%!TEX program = xelatex
% 完整编译: xelatex -> biber/bibtex -> xelatex -> xelatex
\documentclass[lang=cn,12pt,a4paper]{elegantpaper}
\usepackage{pdfpages}
\usepackage{authblk}  
% Added package for spacing
\usepackage{setspace}  
\usepackage{graphicx}

\usepackage[export]{adjustbox}   % 放在导言区
% 设置参考文献样式
\usepackage{fancybox}
\usepackage{tikz}
\usepackage{pdflscape}
% 用于自定义列格式
\usepackage{array} 
% 添加颜色支持宏包
\usepackage{xcolor} 
% 确保浮动格式正常
\usepackage{float}    
% 美化表格线
\usepackage{booktabs} 
% 处理跨行内容
\usepackage{multirow} 
% AMS数学公式支持
\usepackage{amsmath}
% 定理环境
\usepackage{amsthm}
\newcommand{\cornerradius}{0.5cm}
% 定义引理环境
%\newtheorem{lemma}{Lemma}
% 超链接,隐藏链接的颜色框
% 允许使用UTF-8编码
% 使用8位T1编码字体
% 额外的数学符号
% AMS字体
% 自定义列表格式
% 简单的URL排版
\usepackage{url}            
% 文本自动换行和对齐
\usepackage{ragged2e}       
% 自适应列宽的表格
\usepackage{tabularx}       
% 改进浮动对象的处理
\usepackage{float}          
% 含注释的表格
\usepackage{threeparttable} 
% 跨页的长表格
\usepackage{longtable}      
% 自定义标题
\usepackage{caption}        
% 子图
\usepackage{subcaption}     
% 旋转对象
\usepackage{rotating}       
% 附录支持
\usepackage{appendix}       
% 更好的数值和科学记数法
\usepackage{siunitx}        
\title{数字经济}
\author{李展利 \\  中南财经政法大学}
\institute{\href{https://elegantlatex.org/}{Elegant\LaTeX{} 项目组}}

\version{0.10}
\date{}
\hypersetup{
  colorlinks=true,      % 启用彩色超链接
  linkcolor=blue,       % 目录和内部链接为蓝色
  urlcolor=blue,        % URL 链接为蓝色
  filecolor=blue,       % 文件链接为蓝色
  citecolor=blue,       % 文献引用链接为蓝色
}
\definecolor{skillcolor}{RGB}{0,78,150}
\newcommand{\skillexample}[1]{\smallskip\noindent\textcolor{gray}{\textit{\footnotesize\textbf{示例:}} \textit{\footnotesize #1}}}
\newcommand{\lecexample}[1]{\smallskip\noindent\textcolor{gray}{\textit{\footnotesize\textbf{推荐资料:}} \textit{\footnotesize #1}}}
% 本文档命令
\usepackage{array}
\newcommand{\ccr}[1]{\makecell{{\color{#1}\rule{1cm}{1cm}}}}
\usepackage{float}
\begin{document}
\includepdf[pages=-]{论文封面.pdf}
\newpage

\tableofcontents
\newpage  % 目录后分页
\part{报告一:数字经济研究所需知识、技能与数字经济课程设计}
\begin{abstract}
本文概述了数字经济研究所需的跨学科知识(数学、计算机科学、经济学)和核心技能,并设计了一门以培养数字化思维与技术能力为核心的课程。课程涵盖基础理论、关键技术、前沿研究与项目实践,通过多维度评估体系全面考察学生的技术应用和创新能力。
\keywords{数字经济研究 \quad 基础知识 \quad 所需技能 \quad 课程设计}
\end{abstract}
\section{数字经济研究所需知识}
数字经济是全新的经济形态,因此研究数字经济问题需要结合多方面的知识。本文将所需掌握的知识整理成数学、计算机和经济学三大板块,如图~\ref{fig:eco_founda} 所示。本部分安排如下:首先本文在第\ref{sub:math}节介绍所需数学知识;其次,本文在第\ref{sub:cs}节介绍所需计算机知识;最后本文在第\ref{sub:eco}介绍所需经济知识。
\begin{figure}[H]
    \centering
    \includegraphics[width=0.86\linewidth]{intro.png}
    \caption{数字经济研究必备知识}
    \label{fig:eco_founda}
\end{figure}
\subsection{数学基础}
\label{sub:math}
在数字经济研究中,多层次的数学工具既是分析问题的基石,又是构建模型的前提,主要包括代数、分析、概率统计与优化理论等模块。本节安排如图~\ref{fig:math} 所示所示,在介绍相关知识时也会介绍相关应用场景。
\begin{figure}[H]
    \centering
    \includegraphics[width=1\linewidth]{math.png}
    \caption{数学基础}
    \label{fig:math}
\end{figure}
\subsubsection{代数与线性代数}
\begin{itemize}[leftmargin=*,noitemsep]
    \item 熟练运用向量空间与矩阵运算,包括加法、数乘与内积
    \item 深刻理解行列式、逆矩阵和秩的代数性质
    \item 应用场景:主成分分析(PCA)、特征提取等数据降维技术
\end{itemize}
\lecexample{%
\begin{enumerate}[leftmargin=*,nosep]
    \item 丘维声.\textit{高等代数}
    \item 丘维声.\textit{高等代数学习指导书}
\end{enumerate}}
该模块为数字经济中的高维数据降维与特征提取提供了必要的数学支撑,确保算法的准确性与稳定性。

\subsubsection{微积分与数学分析}
\begin{itemize}[leftmargin=*,noitemsep]
    \item 掌握函数极限、连续性,以及一元与多元函数的微分与积分
    \item 理解梯度、方向导数等多元分析概念
    \item 应用场景:优化问题求解、供需动态系统建模
\end{itemize}
\lecexample{%
\begin{enumerate}[leftmargin=*,nosep]
    \item 陈纪修.\textit{数学分析}
    \item 张筑生.\textit{数学分析新讲}
\end{enumerate}}
微积分是研究数字经济动态变化与优化策略的基础,支持模型求解与敏感性分析。

\subsubsection{概率与统计基础}
\begin{itemize}[leftmargin=*,noitemsep]
    \item 掌握期望、方差、协方差等基本统计量
    \item 理解最大似然估计与最小二乘法原理
    \item 应用场景:线性回归、推荐系统中的评分预测
\end{itemize}
\lecexample{%
\begin{enumerate}[leftmargin=*,nosep]
    \item 茆诗松.\textit{概率论与数理统计教程}
    \item Ross S. M.(梁宝生等译).\textit{概率论基础教程}
\end{enumerate}}
该部分提供了数据分析与因果推断的理论基础,是评估数字经济政策与市场行为的理论依据。

\subsubsection{优化理论}
\begin{itemize}[leftmargin=*,noitemsep]
    \item 熟练掌握凸优化、梯度下降与拉格朗日乘子法
    \item 理解牛顿法与黄金分割法等常用数值求解技术
    \item 应用场景:资源配置优化、价格策略设计
\end{itemize}
\lecexample{%
\begin{enumerate}[leftmargin=*,nosep]
    \item 文再文.\textit{最优化:建模、算法与理论}
    \item 袁亚湘.\textit{最优化理论与方法}
\end{enumerate}}
优化理论是制定数字经济中资源配置和定价策略的核心工具,保障决策过程的最优性\footnote{在报告二中运用凸优化知识证明了模型的全局最优存在域。}。

\subsubsection{高等分析理论}
\begin{itemize}[leftmargin=*,noitemsep]
    \item 掌握赋范空间与巴拿赫空间的核心性质
    \item 理解隐函数定理及存在性证明框架
    \item 应用场景:市场均衡分析、动态博弈建模
\end{itemize}
\lecexample{%
\begin{enumerate}[leftmargin=*,nosep]
    \item Rudin W..\textit{Real and Complex Analysis}
    \item Stein E. M..\textit{Real Analysis}
\end{enumerate}}
高等分析为数字经济模型的严谨性与均衡存在性提供了理论保证,是宏观框架分析的基础。

\subsubsection{矩阵理论与分解}
\begin{itemize}[leftmargin=*,noitemsep]
    \item 深入理解奇异值分解(SVD)与特征分解原理
    \item 掌握PCA在数据降维与协方差分析中的应用
    \item 应用场景:交易数据降维、平台匹配算法优化
\end{itemize}
\lecexample{%
\begin{enumerate}[leftmargin=*,nosep]
    \item Horn R. A..\textit{Matrix Analysis}
    \item Golub G. H., Van Loan C. F..\textit{Matrix Computations}
\end{enumerate}}
矩阵分解技术是处理大规模交易与网络数据的核心,支撑平台匹配与推荐系统的高效运行。

\subsubsection{随机过程理论}
\begin{itemize}[leftmargin=*,noitemsep]
    \item 掌握马尔可夫链与泊松过程的建模方法
    \item 理解EM算法与贝叶斯推断的基本思路
    \item 应用场景:用户行为预测、金融风险评估
\end{itemize}
\lecexample{%
\begin{enumerate}[leftmargin=*,nosep]
    \item Durrett R..\textit{Essentials of Stochastic Processes}
    \item Casella G..\textit{Statistical Inference}
\end{enumerate}}
随机过程理论为分析数字经济中不确定性与风险提供了系统框架,是预测与评估的基石。

\subsubsection{现代优化方法}
\begin{itemize}[leftmargin=*,noitemsep]
    \item 熟悉内点法与对偶理论的核心思想
    \item 理解分布式优化框架在大规模系统中的应用
    \item 应用场景:实时竞价系统、多智能体任务调度
\end{itemize}
\lecexample{%
\begin{enumerate}[leftmargin=*,nosep]
    \item Boyd S., Vandenberghe L..\textit{Convex Optimization}
    \item Bertsekas D. P..\textit{Convex Analysis and Optimization}
\end{enumerate}}
该模块适用于支撑大规模在线系统的实时优化与分布式协调,提升数字平台的运行效率\footnote{在报告二中运用全局优化数值算法求解非凸问题。}。

\subsection{计算机技术}
\label{sub:cs}
计算平台与编程工具是数字经济研究的数据处理和模型实现基础,涵盖编程语言、数据处理、系统管理与机器学习框架,如图~\ref{fig:tack} 所示。
\begin{figure}[H]
    \centering
    \includegraphics[width=1\linewidth]{cstack.png}
    \caption{计算机技术}
    \label{fig:tack}
\end{figure}
\subsubsection{编程语言体系}
\begin{itemize}[leftmargin=*,noitemsep]
    \item 核心语言:Python(数据分析)、R(统计建模)、Stata(计量实证)
    \item 辅助工具:Shell 脚本(自动化与运维)
    \item 应用场景:数据清洗、模型实现与批量处理
\end{itemize}
\lecexample{%
\begin{enumerate}[leftmargin=*,nosep]
    \item VanderPlas J..\textit{Python Data Science Handbook}
    \item Kabacoff R. I..\textit{R in Action}
    \item 陈强.\textit{高级计量经济学及 Stata 应用}
    \item Blum R..\textit{Linux 命令行与 Shell 脚本编程大全}
\end{enumerate}}
掌握多种编程语言与脚本技术,可实现大规模数据处理与自动化实验,提升研究效率。

\subsubsection{数据处理技术}
\begin{itemize}[leftmargin=*,noitemsep]
    \item 核心框架:Pandas(数据清洗)、Spark(分布式计算)
    \item 可视化工具:Matplotlib(绘图引擎)
    \item 应用场景:用户行为分析、市场趋势预测
\end{itemize}
\lecexample{%
\begin{enumerate}[leftmargin=*,nosep]
    \item McKinney W..\textit{Python for Data Analysis}
    \item Bruce P..\textit{数据科学实战指南}
    \item 刘思远.\textit{数据可视化:原理与实践}
    \item 李戈.\textit{Spark 大数据处理与分析实战}
\end{enumerate}}
该技术模块是将原始业务数据转化为可分析信息的关键,直接影响模型输入的质量与分析结果的可靠性。

\subsubsection{数据管理系统}
\begin{itemize}[leftmargin=*,noitemsep]
    \item 数据库技术:SQL(关系型)、NoSQL(文档与键值存储)
    \item 工作流工具:Apache Airflow(调度与监控)
    \item 应用场景:高并发交易系统、ETL 管道设计
\end{itemize}
\lecexample{%
\begin{enumerate}[leftmargin=*,nosep]
    \item Forta B..\textit{SQL 必知必会}
    \item 宋天龙.\textit{MongoDB 权威指南与实践}
    \item Housman B..\textit{Data Engineering: Building Reliable Pipelines}
    \item Burkov A..\textit{Data Version Control in Action}
\end{enumerate}}
稳定的数据管理能力为大规模交易与实时分析提供底层保障,是数字平台可靠运行的前提。

\subsubsection{机器学习框架}
\begin{itemize}[leftmargin=*,noitemsep]
    \item 传统算法:Scikit-learn(分类、回归、聚类)
    \item 深度学习:PyTorch(神经网络开发与训练)
    \item 应用场景:用户画像构建、价格弹性预测
\end{itemize}
\lecexample{%
\begin{enumerate}[leftmargin=*,nosep]
    \item 周志华.\textit{机器学习}
    \item Goodfellow I. 等.\textit{Deep Learning}
    \item Jurafsky D., Martin J. H..\textit{Speech and Language Processing}
    \item Szeliski R..\textit{Computer Vision: Algorithms and Applications}
\end{enumerate}}
机器学习框架是实现智能化分析与预测的核心,支撑数字经济中自动化决策与个性化服务\footnote{在报告中利用KAN-LSTM神经网络架构预测未来补货价格和销量。}。

\subsection{经济学理论}
\label{sub:eco}
扎实的经济学理论是构建数字经济分析框架的支撑,有着微观、宏观与计量方法三个维度的扎实基础极大有助于数字经济研究,如图~\ref{eco_tech} 所示。
\begin{figure}[H]
    \centering
    \includegraphics[width=1\linewidth]{eco_fon.png}
    \caption{经济学知识}
    \label{eco_tech}
\end{figure}
\subsubsection{微观经济理论}
\begin{itemize}[leftmargin=*,noitemsep]
    \item 关键概念:网络效应、双边市场、信息不对称
    \item 研究议题:平台竞争策略、数据产权配置
    \item 应用场景:电商平台生态设计、订阅制定价模型
\end{itemize}
\lecexample{%
\begin{enumerate}[leftmargin=*,nosep]
    \item 徐晋.\textit{平台经济学}
    \item Evans D. S., Schmalensee R..\textit{企业触媒策略}
    \item Gawer A..\textit{平台、市场和创新}
    \item 张维迎.\textit{博弈论与信息经济学}
\end{enumerate}}
微观理论为理解平台战略与市场交互机制提供了分析视角,是设计高效商业模式的理论基础。

\subsubsection{宏观经济理论}
\begin{itemize}[leftmargin=*,noitemsep]
    \item 研究重点:数据要素增长模型、数字货币政策
    \item 核心议题:Solow 模型扩展、通吃效应分析
    \item 应用场景:央行数字货币、全要素生产率测算\footnote{数字经济背景下如何融合数字要素重新测算全要素生产率也是一个重要的研究课题,特别是习近平总书记提出新质生产力的背景下。}
\end{itemize}
\lecexample{%
\begin{enumerate}[leftmargin=*,nosep]
    \item 程实、高欣弘.\textit{数字经济与数字货币}
    \item 白津夫、葛红玲.\textit{央行数字货币:理论、实践与影响}
    \item Blanchard O..\textit{宏观经济学}
    \item  Andreas M. Antonopoulos, David A. Harding. \textit{Mastering Bitcoin:
Programming the Open Blockchain}
\end{enumerate}}
宏观理论为评估数字经济对整体经济增长与政策效果提供框架,是制定宏观政策的重要依据。

\subsubsection{计量经济学方法}
\begin{itemize}[leftmargin=*,noitemsep]
    \item 核心技术:面板数据分析、因果推断
    \item 关键方法:DID 设计、GARCH 波动率模型
    \item 应用场景:出口质量研究、加密货币市场分析
\end{itemize}
\lecexample{%
\begin{enumerate}[leftmargin=*,nosep]
    \item 陈强.\textit{高级计量经济学及 Stata 应用}
    \item Ardia D..\textit{GARCH Models: Structure, Statistical Inference and Financial Applications}
    \item Wooldridge J. M..\textit{Introductory Econometrics: A Modern Approach}
\end{enumerate}}
计量方法是验证数字经济模型与实证假设的工具,保障研究结论的可信度与政策建议的有效性。

\newpage
\section {数字经济研究所需技能}\label {sec:skills}
数字经济研究除了需要扎实的基础知识,也需要一系列实用技能,本文总结了相关的重要技能,如图~\ref {fig:tech} 所示。本节首先在第 \ref {sec:problem_identification} 节探讨 “发现问题的能力”,然后在第 \ref {sec:clarify_problem} 节分析 “明确问题的能力”,接着在第 \ref {sec:data_collection} 节阐述 “信息搜集能力”,随后在第 \ref {sec:reporting} 节讨论 “做汇报能力”,再于第 \ref {sec:data_analysis} 节说明 “数据分析能力”,接着在第 \ref {sec:interdisciplinary} 节研究 “跨学科协作能力”,最后在第 \ref {sec:critical_thinking} 节剖析 “批判性思维能力”。在介绍各项能力时,本文从核心能力、重要意义、应用场景和培养路径四个维度展开,同时在每项能力下方提供一个成果示例。\textbf{尽管全面掌握所有能力颇具难度,但认知这些能力的价值并不亚于实践本身。}

\begin {figure}[H]
\centering
\includegraphics [width=1\linewidth]{tech.pdf}
\caption {数字经济研究所需技能}
\label {fig:tech}
\end {figure}

\subsection {发现问题的能力}\label {sec:problem_identification}
\textbf {核心能力:} 作为数字经济研究的逻辑起点,该能力决定了研究方向与潜在价值。研究者需具备敏锐的洞察力,能够在海量信息中精准识别具有研究价值的课题,主要包括:
\begin {itemize}
\item 追踪数字经济前沿动态的能力
\item 对产业变革的敏感度
\item 对社会需求的捕捉能力
\end {itemize}

\textbf {重要意义:} 数字经济的快速演进催生了大量新现象与新问题,研究者需及时捕捉并转化为具体研究议题,如平台垄断、数据隐私、数字鸿沟等。

\textbf {应用场景:}
\begin {itemize}
\item 监测数字支付普及率变化,揭示传统金融与数字金融融合中的痛点
\item 分析用户行为数据,识别数字经济对消费决策的重塑作用
\item 在研究过程中动态捕捉新现象,据此调整研究方向
\end {itemize}

\textbf {培养路径:}
\begin {itemize}
\item 持续关注领域前沿(阅读权威期刊、参加行业会议)
\item 加强跨学科交流,从多元视角发现问题
\item 掌握文献综述方法,通过系统梳理发现研究空白
\item 提升从海量数据中提炼关键信息的能力
\end {itemize}

\skillexample {
\textbf {艾丝特・杜芙若(Esther Duflo)}\\
\textbf {发现:} 在数字经济尚未全面崛起的阶段,杜芙若敏锐察觉到传统经济援助模式在应对贫困问题时的局限性,尤其在数字技术逐步渗透社会经济生活的背景下,传统模式难以适应新环境。她与班纳吉等学者通过严谨的田野实验(RCT)在印度等地揭示了贫困成因的复杂性,指出传统援助项目设计常忽视受助者的行为与心理倾向(如在营养与口味间的权衡)。在数字经济时代,这种对人的行为和心理的研究,为设计数字普惠金融产品、数字扶贫项目等提供了重要启示,助力研究者探索让数字经济更好服务弱势群体的路径。\\
\textbf {贡献:} 运用随机对照试验,将受试群体分为对照组与实验组,精准测量不同干预措施(如小额信贷、驱虫药、绩效工资)对健康与生产力的影响,进而发现 “小而有效” 的扶贫策略,推动经济学研究方法向微观实证化发展。这种实证研究方法为数字经济研究提供了范式,研究者可借鉴其思路,在数字经济领域通过实验验证新商业模式、政策措施等的有效性,如验证数字广告投放策略对消费者行为的影响,推动数字经济研究走向更科学精准的道路。
}

\subsection {明确问题的能力}\label {sec:clarify_problem}
\textbf {核心能力:} 将模糊研究方向转化为清晰课题是研究成功的关键,包括:
\begin {itemize}
\item 精准表述研究问题
\item 清晰定义关键概念
\item 合理界定研究范围
\end {itemize}

\textbf {重要意义:} 数字经济问题具有跨领域性且边界模糊,需准确界定研究对象与范畴,避免概念混淆。

\textbf {应用场景:}
\begin {itemize}
\item 确定 “数字鸿沟” 研究侧重技术接入、技能差距或信息获取差异
\item 在 “平台经济” 研究中明确讨论商业模式、监管框架或社会影响
\item 根据新兴现象动态调整研究问题定义
\end {itemize}

\textbf {培养路径:}
\begin {itemize}
\item 学习概念分析与理论建构方法
\item 使用研究问题模板(如 “如何……”“为何……” 等)
\item 通过练习与同行评审提升表述准确度
\item 学会根据实证结果动态修正问题边界
\end {itemize}

\skillexample {
\textbf {罗纳德・科斯(Ronald Coase)}\\
\textbf {问题:} 在数字经济蓬勃发展的当下,外部性问题愈发复杂,如数据泄露的社会影响、数字平台垄断的负外部性等。科斯观察到外部性问题并非单纯的市场失灵,而是产权界定与交易成本的综合体现,这一观点在数字经济时代依然适用,数据产权模糊、数字交易成本高等问题亟待明确界定与解决。\\
\textbf {贡献:} 在《社会成本问题》中,他将 “外部性冲突” 转化为 “产权不清与交易成本高昂” 问题,提出科斯定理:若交易成本为零,市场自主谈判总能实现资源最优配置,从而奠定新制度经济学基础。在数字经济研究中,研究者可借鉴科斯明确问题的思路,清晰界定研究问题,如在研究数字版权问题时,明确版权归属、使用范围等,为数字经济领域的制度设计与政策制定提供理论支撑,推动数字经济在清晰规则框架下健康发展。
}

\subsection {信息搜集能力}\label {sec:data_collection}
\textbf {核心能力:} 数据搜集是研究的基础,决定了研究结果的广度与深度,包括:
\begin {itemize}
\item 学术文献检索
\item 行业报告收集
\item 企业与政府数据获取
\item 社交媒体与大数据采集
\end {itemize}

\textbf {重要意义:} 研究高度依赖数据,约 70\% 的时间用于数据获取与预处理,数据质量直接影响结论可信度。

\textbf {应用场景:}
\begin {itemize}
\item 利用爬虫技术抓取电商用户行为数据
\item 调取政府统计数据库中的数字经济规模指标
\item 通过访谈获取企业数字化转型的一手资料
\item 处理非结构化数据,如文本与图像
\end {itemize}

\textbf {培养路径:}
\begin {itemize}
\item 掌握学术与行业数据库检索技巧
\item 学习 API 调用、爬虫开发与问卷设计
\item 熟练运用 Excel、Python Pandas 等工具
\item 理解数据伦理与法律边界
\end {itemize}

\skillexample {
\textbf {安格斯・迪顿(Angus Deaton)}\\
\textbf {问题:} 在数字经济浪潮下,消费模式与福利状况发生显著变化,传统数据搜集方法难以全面反映这些变化。迪顿在消费与福利研究中整合家庭调查、宏观统计与市场价格数据,识别出发展中国家贫困测度与消费模式间的偏差。在数字经济研究中,研究者需类似地整合多源数据,如结合电商平台交易数据、社交媒体数据与政府统计数据,以全面了解数字经济对消费和福利的影响。\\
\textbf {贡献:} 他综合运用多源微观数据与宏观面板数据,提出更精确的消费函数模型,改进了福利政策效果评估方法,并因此获得 2015 年诺贝尔经济学奖。这种数据搜集与整合方法为数字经济研究提供了借鉴,研究者可通过搜集多样化数据,包括数字平台用户行为数据、企业运营数据等,构建更准确的模型,分析数字经济发展趋势,为政策制定与企业决策提供有力依据,推动数字经济研究在数据驱动下不断深入。
}

\subsection {做汇报能力}\label {sec:reporting}
\textbf {核心能力:}
\begin {itemize}
\item 信息结构化与逻辑梳理
\item 演讲技巧与情感感染
\item 数据可视化与多媒体应用
\item 听众互动与问题应对
\end {itemize}

\textbf {重要意义:} 高质量的汇报可扩大研究影响,促进跨界合作与决策采纳。

\textbf {应用场景:}
\begin {itemize}
\item 在学术会议上发表研究成果并回应质询
\item 在团队内部进行进展报告,确保共识
\item 向政策制定者或企业管理层推介研究发现
\item 运用 PowerPoint、Prezi 等工具强化演示效果
\end {itemize}

\textbf {培养路径:}
\begin {itemize}
\item 参加演讲培训,提升现场表达能力
\item 定期进行汇报演练,优化时间与节奏把控
\item 学习高级可视化工具,增强数据说服力
\item 邀请同行点评,迭代改进汇报内容
\end {itemize}

\skillexample {
\textbf {保罗・克鲁格曼(Paul Krugman)}\\
\textbf {问题:} 2008 年金融危机期间,信息繁杂、解读不一,公众与政策层面缺乏清晰分析框架。在数字经济领域,同样存在大量复杂信息,如数字金融风险、平台经济发展趋势等,如何向不同受众清晰传达研究成果成为关键问题。\\
\textbf {贡献:} 克鲁格曼通过《纽约时报》博客与公开演讲,将复杂的宏观经济学原理与实证数据结合,以生动语言和图表解析危机成因与应对方案,显著提升了学术界与大众的理解度,并因撰写通俗经济专栏获得 2008 年诺贝尔经济学奖。在数字经济研究汇报中,研究者可借鉴克鲁格曼的方法,运用数据可视化工具,如制作动态图表展示数字经济指标变化趋势,用通俗易懂的语言向政策制定者阐述数字经济监管策略,向企业管理者讲解数字营销方案,从而有效扩大研究影响力,促进数字经济研究成果的应用与推广。
}

\subsection {数据分析能力}\label {sec:data_analysis}
\textbf {核心能力:}
\begin {itemize}
\item 数据预处理与清洗
\item 可视化展示
\item 建模与预测
\item 统计推断与因果识别
\end {itemize}

\textbf {重要意义:} 描述性分析已难以满足研究需求,需借助深度模型与因果推断揭示作用机制。

\textbf {应用场景:}
\begin {itemize}
\item 回归分析数字支付普及对 GDP 增长的影响
\item 聚类识别不同消费群体的行为特征
\item 时间序列预测数字经济发展趋势
\item 应用机器学习优化电商推荐系统
\end {itemize}

\textbf {培养路径:}
\begin {itemize}
\item 学习假设检验、回归分析等统计理论
\item 掌握 R、Python、SPSS 等分析工具
\item 精通 Matplotlib、Tableau 等可视化软件
\item 深入理解聚类、分类、回归等机器学习算法
\end {itemize}

\skillexample {
\textbf {哈尔・瓦里安(Hal Varian)}\\
\textbf {问题:} 在互联网时代,如何运用海量数据预测用户行为与市场动态?在数字经济蓬勃发展的今天,这一问题更为突出,如电商平台需预测用户购买行为以优化供应链,社交媒体平台要预测用户兴趣以精准推送内容。\\
\textbf {贡献:} 瓦里安结合经济学与机器学习方法,提出信息经济学模型,并在 Google 担任首席经济学家期间,领导团队利用点击流数据与拍卖模型优化在线广告定价,实现了数十亿美元的收入提升。这种数据分析方法为数字经济研究提供了范例,研究者可运用机器学习算法分析用户在数字平台上的大量行为数据,建立预测模型,如预测数字产品的市场需求、用户对数字服务的满意度等,通过深入的数据分析与模型构建,为数字经济企业的决策提供科学依据,推动数字经济在数据驱动的分析中实现创新发展。
}

\subsection {跨学科协作能力}\label {sec:interdisciplinary}
\textbf {核心能力:}
\begin {itemize}
\item 跨学科知识整合
\item 团队协作与项目管理
\item 与非学术机构的沟通
\end {itemize}

\textbf {重要意义:} 技术、法律、经济与社会因素交织,单一学科难以全面把握数字经济问题。

\textbf {应用场景:}
\begin {itemize}
\item 与计算机科学家研究区块链在支付系统中的应用
\item 与法学专家探讨数据隐私与监管框架
\item 与管理学者共研企业数字化转型战略
\item 参与国际联合项目,比较不同国家的数字经济实践
\end {itemize}

\textbf {培养路径:}
\begin {itemize}
\item 参加跨学科研讨会与工作坊
\item 在联合项目中担任协调者或技术顾问
\item 学习将专业术语转化为通用语言
\item 培养团队管理与冲突解决能力
\end {itemize}

\skillexample {
\textbf {约瑟夫・斯蒂格利茨(Joseph Stiglitz)}\\
\textbf {问题:} 信息不对称不仅是市场失灵问题,更涉及法律与社会制度设计。在数字经济领域,信息不对称问题更为严峻,如数据垄断企业掌握大量用户数据,而用户却缺乏对自身数据的掌控权,同时相关法律与制度尚不完善。\\
\textbf {贡献:} 斯蒂格利茨将经济学、法学与政治学相结合,提出信息不对称理论,揭示市场机制中信息分配不均导致的效率损失,为监管政策与制度设计提供了多学科视角,并于 2001 年获得诺贝尔经济学奖。在数字经济研究中,研究者可借鉴这种跨学科协作方式,联合经济学家、计算机科学家、法律专家等,共同研究数字经济中的复杂问题,如探讨如何运用技术手段保障数据安全、制定合理法律规范数据使用、从经济角度分析数据价值等,通过跨学科的知识整合与团队协作,推动数字经济在完善的制度和技术支撑下健康有序发展。
}

\subsection {批判性思维能力}\label {sec:critical_thinking}
\textbf {核心能力:} 批判性思维助力研究者深入分析既有理论与数据,包括:
\begin {itemize}
\item 评估文献可信度与局限性
\item 提出替代理论与反驳观点
\item 从多角度审视问题
\end {itemize}

\textbf {重要意义:} 在快速发展的领域中,需质疑传统观点以推动理论创新。

\textbf {应用场景:}
\begin {itemize}
\item 对 “平台经济” 理论进行批判性分析,提出 “平台生态系统” 模型
\item 对 “数字货币” 研究质疑其安全性和应用边界
\item 在数据分析中识别偏差并设计修正方案
\end {itemize}

\textbf {培养路径:}
\begin {itemize}
\item 阅读经典批判性文献,学习质疑与分析技巧
\item 在写作中反复推敲论点并提出备选假设
\item 参与学术讨论,接受对自身观点的挑战
\item 与同行进行思维碰撞,激发创新灵感
\end {itemize}

\skillexample {
\textbf {阿马蒂亚・森(Amartya Sen)}\\
\textbf {问题:} 传统福利经济学以效用为核心,忽视了个体能力与自由的重要性。在数字经济时代,传统经济理论和评价体系面临诸多挑战,如数字经济创造的价值如何衡量,数字技术对个体能力和自由的影响等问题,传统理论难以全面解答。\\
\textbf {贡献:} 森提出 “能力” 视角,强调衡量社会福祉应关注人们的实际能力与选择自由,颠覆了单一效用指标的方法论框架,为发展经济学与公共政策研究注入全新批判性视野,并获得 1998 年诺贝尔经济学奖。在数字经济研究中,研究者应秉持批判性思维,对现有的数字经济理论和实践进行反思与质疑,如重新审视数字平台的价值创造模式、评估数字技术对社会公平的影响等,通过批判性思考提出新的理论和解决方案,推动数字经济理论的创新与发展,使数字经济更好地服务于社会福祉的提升。
}
\section{数字经济课程设计方案}
本节安排如下:首先本文在第\ref{sec:history}节梳理了数字经济课程建设沿革,回顾了数字经济课程从早期的学科孕育期到当前的系统重构期的发展历程。在此基础上,本文进一步介绍了数字经济课程的核心组成部分,具体内容可参见第\ref{sec:overview}节。在后续的\ref{sec:modules}节中,我详细阐述了课程的核心模块及其对应的学习目标与方法。此外,本文还在第\ref{sec:goals}节中明确了本课程的学习目标,包括认知目标、能力目标和价值观目标,为学生提供了清晰的学习路径和发展方向。最后,课程的考核方案在第\ref{sec:assessment}节进行了详细说明,确保学生能够通过综合性的评价体系全面掌握课程内容并应用于实际项目中。课程设计思维导图如图~\ref{fig:intro_course} 所示。
\begin{figure}[H]
    \centering
    \includegraphics[width=1\linewidth]{exported_image.png}
    \caption{课程设计概览}
    \label{fig:intro_course}
\end{figure}


\subsection{数字经济课程历史沿革} \label{sec:history}
正如图~\ref{fig:dig_eco} 所示,数字经济课程的发展可分为三个阶段:学科孕育期(1950年代-1999),课程起源于计算机科学与信息经济学的交叉地带,因技术割裂与理论局限,尚未形成系统认知框架;融合探索期(2000-2015),互联网商业化引发课程范式迁移,虽出现教学尝试,但整体存在教材更新滞后、治理实践脱节、本土化不足等问题;系统重构期(2016年至今),AI与数据要素引领课程从“经济学+ICT”走向“技术-经济-治理”三维融合,产教协同机制趋于成熟\footnote{ICT即为Information and Communication Technology(信息与通信技术),指集成信息技术与通信技术的综合技术领域}。整体来看,课程体系演进受“技术适配度”与“制度嵌入性”双因素驱动,随着BEA定义、国家数据局等制度创新的落地,课程迭代周期已从过去的5-7年缩短至18个月,标志着全球数字经济教育进入高频更新、深度融合的新阶段\footnote{Bureau of Economic Analysis(美国经济分析局,缩写BEA)是美国商务部下属机构,负责发布国民经济核心数据(如GDP、贸易收支、数字经济卫星账户等),为政策制定提供依据,其定义的数字经济统计框架(如数字服务分类、数据资产核算)是课程“制度嵌入性”的典型体现}。
\begin{figure}[H]
    \centering
    \includegraphics[width=1\linewidth]{his_dig.png}
    \caption{数字经济三大阶段}
    \label{fig:dig_eco}
\end{figure}

\subsection{课程概述} \label{sec:overview}
本课程以“数字经济”为纲,贯彻“懂技术、会管理”的教学理念,核心聚焦于数字经济的关键支撑技术,特别是机器学习和深度学习等人工智能技术如何与具体的数字经济问题相结合。

当前,“数字化转型”、“智慧金融”、“平台经济”等研究热潮方兴未艾,但也存在部分流于概念、换汤不换药的现象。突破此局限的关键,在于深入把握支撑数字经济的底层技术体系。例如,评估企业数字化水平,绝不能停留于年报关键词的简单统计,必须深入理解其背后的大数据平台、云计算架构与人工智能技术栈的作用;不仅要会获取数据,更要能结合具体业务场景。唯有如此,学生才能真正理解数字经济的运行机理,并具备驱动其落地应用的实战能力。

理论与实践的深度交融是本课程的基石。为此,课程将约50\%的学时聚焦于数字经济核心技术的学习与实践,重点夯实机器学习与深度学习等关键方法。

\begin{figure}[H]
    \centering
    \includegraphics[width=1\linewidth]{course.pdf}
    \caption{课程核心组成}
    \label{fig:course}
\end{figure}

\subsection{课程目标} \label{sec:goals}
通过本课程的学习,学生将能够:
\begin{itemize}
    \item \textbf{认知目标:}
    \begin{enumerate}
        \item 深入理解数字经济的核心概念,掌握其定义、发展历程、核心思想及在全球产业格局中的重要角色。
        \item 理解支撑数字经济发展的关键技术原理,包括机器学习、深度学习、云计算和大数据平台的基础。
        \item 熟悉数字经济的典型应用场景,并掌握相关的政策法规与治理框架。
    \end{enumerate}
    \item \textbf{能力目标:}
    \begin{enumerate}
        \item 能够灵活运用计量经济学与机器学习方法,分析与解决数字经济中的实际问题。
        \item 掌握数据获取、预处理及探索性分析的技能,能够使用Python、Hadoop、Spark等工具进行数据处理。
        \item 通过项目实践,设计并实现基于实际业务背景的解决方案,包括技术实现和报告撰写。
        \item 培养团队合作与项目管理的能力,能够在团队中高效协作,共同完成期末项目。
    \end{enumerate}
    \item \textbf{价值观目标:}
    \begin{enumerate}
        \item 培养数字化思维,能够从技术角度深入理解数字经济的本质及其影响。
        \item 激发创新能力,在解决实际问题时体现批判性思维和创造性。
        \item 理解数字经济对社会的深远影响,培养职业道德和社会责任感。
    \end{enumerate}
\end{itemize}

\subsection{课程核心模块} \label{sec:modules}
本课程共设四个模块,旨在全面提升学生在数字经济领域的理解与实践能力。课程内容首先从基础概念入手,逐步引导学生掌握数字经济的核心理念;然后深入探讨支撑数字经济的关键技术;接着展示前沿的研究成果与行业发展动态;最后通过实际项目实践,提升学生的操作能力与项目管理能力。

\subsubsection{模块一:数字经济基础(第1--3周)} \label{sec:module1}
此模块帮助学生全面理解数字经济的基本概念及其发展背景。
\begin{itemize}
    \item \textbf{第1周:传统经济与数字经济的演进对比}:讨论传统经济的特点及数字经济的崛起,分析二者之间的异同与联系。学生将学习到数字经济如何通过技术创新突破传统经济的局限。
    \item \textbf{第2周:数据、算法、平台与网络的协同作用}:深入讲解数据、算法、平台和网络如何在数字经济中相互作用,强调数据驱动经济活动的核心作用。
    \item \textbf{第3周:数字经济在全球产业格局中的角色}:分析全球数字经济的发展趋势,探讨不同国家如何通过政策引导与技术创新,推动数字经济的全球布局。
\end{itemize}

\subsubsection{模块二:关键技术与工具(第4--12周)} \label{sec:module2}
本模块着重讲解支撑数字经济的核心技术,并通过案例分析提升学生的技术应用能力。
\begin{itemize}
    \item \textbf{第4周:计量经济学基本方法与机器学习模型的差异}:讲解计量经济学与机器学习模型的基本区别,帮助学生了解两者的应用场景与技术基础。
    \item \textbf{第5周:广义线性模型(Lasso、Ridge、Elastic Net)}:深入探讨广义线性模型的基本原理,学习如何使用Lasso、Ridge、Elastic Net等方法进行变量选择与模型优化。
    \item \textbf{第6--7周:树模型与集成算法(决策树、随机森林、GBDT、XGBoost)}:讲解树模型及其集成方法,包括决策树、随机森林、GBDT和XGBoost,重点培养学生在实际数据分析中运用这些技术的能力。
    \item \textbf{第8--9周:神经网络(多层感知机、卷积神经网络、循环神经网络)}:通过深入讲解多层感知机、卷积神经网络(CNN)和循环神经网络(RNN),帮助学生掌握神经网络在图像、语音和时间序列分析中的应用。
    \item \textbf{第10--12周:云计算与大数据平台(Hadoop、Spark)}:介绍云计算与大数据技术,包括Hadoop与Spark平台的应用,学生将通过实践操作,掌握如何处理与分析大规模数据集。
\end{itemize}

\subsubsection{模块三:前沿文献与行业洞察(第13--14周)} \label{sec:module3}
本模块通过专题讲座与案例分析,帮助学生深入了解数字经济中的前沿研究与行业发展趋势。
\begin{itemize}
    \item \textbf{第13周:博弈论在数字经济中的应用}:通过案例分析,介绍现代计算技术在复杂博弈模型求解中的应用,帮助学生理解博弈论在数字经济中的重要性\citep{sandholm2010state,hu2023recent}。
    \item \textbf{第13周:双重机器学习在微观实证中的应用}:讲解双重机器学习在微观经济学实证分析中的应用,培养学生使用新型方法分析经济问题的能力\citep{chernozhukov2018double,athey2019machine}。
    \item \textbf{第13周:深度神经网络在宏观经济学中的应用}:介绍深度神经网络在时间序列预测中的应用,特别是如何利用深度学习技术预测宏观经济指标\citep{lim2021time,benidis2022deep}。
    \item \textbf{第14周:大语言模型(LLM)在市场营销中的应用}:分析大语言模型(如GPT-4)在市场感知调查与消费者行为预测中的应用\citep{wang2024large,arora2025ai}。
    \item \textbf{第14周:运筹优化中的数据驱动方法}:介绍数据驱动的学习优化和优化学习范式,帮助学生理解数据如何推动运筹优化的进步\citep{andrychowicz2016learning,chen2022learning}。
    \item \textbf{第14周:人工智能在金融市场中的应用}:探讨人工智能技术(如强化学习和深度学习)在金融市场中的应用,特别是在投资策略优化和风险管理中的应用\citep{huang2024multi,yang2024finrobot}。
\end{itemize}

\subsubsection{模块四:项目实践与应用(第15--16周)} \label{sec:module4}
本模块通过实践项目,帮助学生巩固所学知识,提升团队合作与项目管理能力。
\begin{itemize}
    \item \textbf{第15周:结课开题报告}:各小组需提交项目计划书并进行开题答辩,系统阐述项目目标、技术路线与预期成果。
    \item \textbf{第16周:项目成果报告}:学生将撰写技术报告,并进行项目展示与答辩,提升项目汇报与沟通能力。
\end{itemize}

通过本模块的学习,学生不仅能应用所学的技术知识,还能提升团队合作能力和项目管理能力,为未来的职业生涯打下坚实基础。

\subsection{课程评分规则} \label{sec:assessment}
表 \ref{tab:assessment} 展示了本课程的评分分布,其中加分项的最高分为10分,最终课程总分为100分,旨在全面评估学生的学习成果和实践能力。
\begin{table}[H]
  \centering
    \caption{数字经济课程考核方案}
  \begin{tabular}{p{4cm} >{\centering\arraybackslash}p{2cm} >{\centering\arraybackslash}p{8cm}}
    \toprule
    考核环节 & 权重(\%) & 说明 \\
    \midrule
课堂参与与出勤 & 20 & 包括出勤、课堂讨论、案例分享等;鼓励主动提问与团队协作。 \\
期中开放式开卷考试 & 10 & 考试内容为开放式作答,要求学生自主写出想要研究的数字经济问题(只需要给出大致框架思路),限时2小时,为期末报告做准备,提醒学生自主学习,开卷考试突出强调学生查阅资料解决问题的能力。 \\
课后作业 & 30 & 对应模块二(第4–12周)中的动手实践,每次作业需提交代码与报告。 \\
期末项目实践 & 40 & 包括:方案设计与数据分析(15\%)、技术实现与报告(10\%)和团队汇报与答辩(15\%) \\
\textcolor{blue}{加分项} & \textcolor{blue}{10} & \textcolor{blue}{课堂回答问题、期末项目出彩}\\
    \bottomrule
  \end{tabular}
  \label{tab:assessment}
\end{table}

\newpage
\part{报告二:数字化农产品补货与定价策略:深度学习算法设计与实证验证}
\begin{abstract}
数字经济的发展离不开农业供应链管理数字化水平的提升。本文提出了一个数据驱动框架,集成KAN-LSTM模型与非线性规划方法,以优化农产品市场中新鲜蔬菜的采购和定价决策。在预测阶段,KAN-LSTM模型在需求预测和补货价格预测方面展现出卓越性能,其消融实验效果显著优于LSTM和xLSTM基准模型。KAN-LSTM的最佳预测指标为:$R^2=0.9903$、$RMSE=2.4215$、$MAPE=2.1889\%$。基于此预测结果,多元逐步回归模型发现生鲜农产品大多为投机商品。在定价优化方面,本文证明了当需求价格弹性$\beta_i$满足$\beta_i \leq -1$或$\beta_i \geq 0$时存在全局最优解。同时在非凸优化求解中,模拟退火算法通过全局优化15天规划期内的采购量和加价率,实现了$12,881.68$元的总利润。该数字化框架可显著提升农产品市场的运营效率与资源配置有效性。相关数据集、代码及附件已在\url{https://github.com/Zhanli-Li/Digital-Agricultural-Economics}开源。
\keywords{数据驱动框架 \quad KAN-LSTM \quad 农业供应链 \quad 非凸优化}
\end{abstract}
\setcounter{section}{0}
\renewcommand{\theHsection}{partB.\arabic{section}} % 添加唯一前缀
\section{引言}
\label{sec:introduction}
在数字经济时代,发展数字化农业供应链至关重要,本文立足数字经济现实问题,提出了一个数据驱动的框架,预期提升农业供应链数字化水平进而促进我国数字经济建设。本节安排如下:首先,我在第\ref{subsec:policy_background}节介绍政策背景;其次,我在第\ref{subsec:theoretical_background}节阐述理论背景;第三,我在第\ref{subsec:real_background}节描述现实背景;第四,我在第\ref{subsec:problem_definition}节定义研究问题;最后,我在第\ref{subsec:research_contribution}节总结研究贡献。

\subsection{政策背景}
\label{subsec:policy_background}
% ---------- 历史沿革 ----------
自古以农立国,农强则国强,农安则天下安。中华民族自古以来对于农业供应链发展极为重视:春秋战国时期,齐相管仲推行"平准仓法",通过国家储备调节粮价保障供给,开创系统化治理先河。李冰修建都江堰后,"使蜀沃野千里,号为陆海",使成都平原成为秦汉帝国坚实的"后方粮仓"。隋唐大运河的开凿构筑庞大水运网络,"使江浙之粟可渐达于涿郡",洛阳含嘉仓的巨型窖藏见证了中央集权下的供应链效能。宋代则达致精细化管理巅峰:王安石的"青苗法"与"市易法"分别保障生产源头供给与流通环节调控,配合标准化漕运及徽商构建的民间粮贸网络,共同维系庞大帝国的运转。

进入21世纪,伴随科技进步与数字经济高速发展,传统农业正经历深刻变革。区别于既往经验驱动的农业供应链管理,现代农业供应链正转向数据驱动的智能化模式。在数字经济与人工智能蓬勃发展的当下,如何实现农业供应链的数智化管理已成为国家战略层面的重要议题。
\begin{figure}
    \centering
  % roundcorner=5pt 指定圆角半径
  \includegraphics[width=\linewidth]{policy.pdf}
\end{figure}
% ---------- 政策演进 ----------
2017年国务院办公厅印发\href{https://www.gov.cn/gongbao/content/2017/content_5234516.htm}{《关于积极推进供应链创新与应用的指导意见》},首次将农业供应链纳入国家战略框架\footnote{https://www.gov.cn/gongbao/content/2017/content\_5234516.htm}。明确提出"推动农村一二三产业融合发展",鼓励建立集生产、加工、流通和服务于一体的农业供应链体系。设定2020年形成覆盖重点产业的智慧供应链体系,培育约100家全球领先供应链创新企业等政策目标。此阶段政策侧重于供应链组织形式创新与基础能力建设,为后续数据驱动的农业供应链发展奠定基础。

2019年农业农村部会同中央网信办联合发布\href{https://www.gov.cn/zhengce/zhengceku/2020-01/20/content_5470944.htm}{《数字农业农村发展规划(2019-2025年)》},标志着农业供应链管理进入数字化转型阶段\footnote{https://www.gov.cn/zhengce/zhengceku/2020-01/20/content\_5470944.htm}。规划明确"数据作为关键生产要素"的定位,强调基础数据资源体系建设,强化数字化生产能力,加快农业农村生产经营管理服务数字化改造。设定2025年农业数字经济占比达15\%、农产品网络零售额占比达15\%、农村互联网普及率达70\%等具体指标,为农业供应链数智化转型提供明确指引与量化标准。

2020年农业农村部等六部门印发\href{https://www.gov.cn/zhengce/zhengceku/2020-05/08/content_5509715.htm}{《"互联网+"农产品出村进城工程试点工作方案》},着力构建农产品产销一体化供应链\footnote{https://www.gov.cn/zhengce/zhengceku/2020-05/08/content\_5509715.htm}。提出2020年底前每个农产品生产县培育1-3个特色优质农产品品牌,至少形成1条产销一体的农产品电商供应链,推动"互联网+"与农村一二三产业深度融合。此阶段政策重心转向数字技术在农业供应链各环节的具体应用,特别聚焦电商平台与农产品供应链的融合创新。

2024年农业农村部发布\href{https://www.gov.cn/zhengce/zhengceku/202410/content_6983051.htm}{《农业农村部关于大力发展智慧农业的指导意见》},进一步细化数据驱动型农业供应链建设路径\footnote{https://www.gov.cn/zhengce/zhengceku/202410/content\_6983051.htm}。文件强调"坚持数据驱动、普惠共享",主张通过资源整合与数据共享促进数据融合、挖掘与应用;构建共享平台实现农业务互联、资源共建共享、业务协同。提出"先易后难、循序渐进"的实施策略:优先选择基础较好区域、关键领域和重要品种,按"一品种一链条"方式开展大数据试点,边试点边总结,稳步实现成效可视化。

2025年商务部等八部门联合印发\href{https://www.mofcom.gov.cn/zwgk/zcfb/art/2025/art_7db50f28395b49e9851fb27e4d2c1aed.html}{《加快推进数智供应链创新发展专项行动方案》},标志着农业供应链管理进入数智化转型阶段\footnote{https://www.mofcom.gov.cn/zwgk/zcfb/art/2025/art\_7db50f28395b49e9851fb27e4d2c1aed.html}。方案明确设定2030年形成可复制可推广的数智供应链建设发展模式,在重点行业关键领域基本建成深度嵌入、智能高效、自主可控的数智供应链体系,培育约100家全国供应链数智化领先企业等目标。农业领域要求发展智慧农业,深化"互联网+"农产品出村进城工程,推动农业全领域全环节数智化转型,为农业数智供应链发展夯实基础。

随着科技进步和数字化浪潮的不断推进,农业供应链管理正从传统的经验驱动逐步向数据驱动、智能化管理模式转型。各级政府出台的政策文件,从加强基础能力建设到推动数字化与智慧农业的深度融合,为农业供应链的现代化奠定了坚实基础。展望未来,随着5G、物联网、大数据与人工智能等新技术的持续应用,农业供应链的数字化、智能化进程将进一步加速,推动农业全产业链的效率提升和质量保障,助力实现可持续发展目标。农业供应链的创新与变革,必将成为国家粮食安全、农村振兴和经济高质量发展的重要支撑。
\subsection{理论背景}
\label{subsec:theoretical_background}

需求预测与价格优化构成现代零售运营的核心决策模块,二者的协同优化直接决定库存周转效率与利润最大化目标的实现路径 \citep{Mahapatra2025,Lu2024}。在生鲜领域,由于易腐性、季节性及受天气、促销等外生变量的显著影响,精准捕捉需求动态并制定适应性定价策略已成为跨学科前沿课题 \citep{Badakhshan2024,Hashemi-Amiri2023,Hofmann2018}。

长短期记忆网络(LSTM)作为时间序列预测的主流工具,其通过门控机制有效建模长期依赖关系 \citep{Hochreiter1997}。但传统LSTM处理生鲜需求数据时面临挑战:短期波动特征与长期趋势特征高度耦合,特征空间存在复杂非线性映射。原生LSTM的单记忆细胞结构难以动态调整不同时间尺度的特征提取权重,导致多尺度特征融合效率不足 \citep{Chien2021}。鉴于此,本文通过引入分层注意力机制构建改进型KAN-LSTM模型:在保留原生LSTM记忆细胞基础上,增加短期波动增强层与长期趋势精炼层,实现对生鲜蔬菜销量-进价时序多尺度特征的差异化建模,从根本上提升预测模型的时变适应性。

价格弹性理论体系为定价决策提供理论基础,包含两个关键维度:自身价格弹性与交叉价格弹性 \citep{Ellison2009}。前者量化需求对自身价格变动的敏感性,后者描述相关产品价格波动对目标产品需求的交叉影响 \citep{Mantrala2006,Chen2023}。传统弹性估计计量模型(如OLS回归)存在显著局限:其线性假设框架难以捕捉产品间复杂非线性交互作用,在高维变量空间易出现多重共线性问题 \citep{Rushchitsky2016,Bell2019}。非参数模型(如神经网络)虽在非线性关系建模上展现优势,但其"黑箱"特性导致弹性估计缺乏经济学可解释性,难以直接服务实际定价决策。本文采用基于逐步回归技术的多元线性回归模型:通过双向变量选择机制(后向消除与前向选择结合)动态优化变量集,在控制多重共线性同时有效识别具显著经济意义的弹性系数,为构建包含价格-需求联动关系的定价库存决策模型提供可解释理论框架 \citep{Kamakura1996,Walters1996}。

在多约束非线性优化领域,算法选择需权衡解空间搜索能力、计算复杂度与约束适应性 \citep{Belkourchia2019}。序列最小二乘二次规划(SLSQP)作为典型梯度基方法,处理连续可微约束时收敛速度快,但其局部搜索特性在非凸问题中易陷入局部最优 \citep{Gad2022a}。相较而言,粒子群优化(PSO)、遗传算法(GA)、模拟退火(SA)等元启发式算法通过自然现象模拟构建全局搜索机制:PSO利用粒子间信息共享与协作在高维空间展现高效局部搜索能力 \citep{Gad2022b};GA通过选择-交叉-变异操作在离散变量优化问题中实现强全局探索性 \citep{Mangla2021};SA模拟物理系统退火过程,采用Metropolis准则概率性逃离局部最优,具有独特的搜索深度与广度平衡特性 \citep{Yunzhu2023a}。针对生鲜零售场景下订货量与加价优化的混合整数非线性规划问题,本文选择SA算法作为求解工具。其优势在于设计自适应温度衰减函数与邻域搜索策略:在保证求解质量同时将计算复杂度控制在多项式时间内,且较梯度基方法在处理非光滑目标函数时展现更强鲁棒性。

然而现有研究在以下三方面仍有改进空间:首先,主流预测模型在区分生鲜需求的短期波动与长期趋势时效果有限,尚难以充分应对市场的动态变化;其次,弹性系数的估计方法往往在模型复杂度与经济学可解释性之间难以兼顾,使得价格决策缺乏更可靠的实证支持;再次,多周期优化算法在全局搜索效率与解的质量之间尚未找到理想平衡,可能导致供应链决策响应滞后。鉴于此,本文尝试构建一个集成时序预测、弹性参数估计与全局决策优化的协同框架:我们在分层注意力机制基础上,对 KAN-LSTM 进行了适当调整,以更好地捕捉多尺度特征;引入了一种动态变量选择的回归方法,以期提高弹性系数估计的稳健性与可解释性;并基于模拟退火思路,提出一种自适应调节策略,力图在非凸优化中兼顾收敛性和解的质量。
\subsection{现实背景}
\label{subsec:real_background}
农产品供应链的高效、稳定运行,不仅关乎亿万农民的生计与消费者的“菜篮子”安全,更是国家粮食安全和乡村振兴战略的基石。然而,传统农产品供应链,尤其是生鲜农产品领域,正面临着深刻的结构性挑战。

\textbf{信息孤岛与决策失效:}供应链各环节(生产端、批发端、零售端)信息割裂严重,形成典型的“信息孤岛”。生产者(农户)往往基于经验或有限信息决策种植品类和规模,对瞬息万变的市场需求感知滞后(“种什么”、“种多少”的盲目性)。批发商和零售商则因缺乏精准的需求预测和历史数据支持,在“订什么”、“订多少”、“定什么价”等核心决策上高度依赖主观判断,导致“牛鞭效应”放大:上游微小的需求波动引发下游巨大的库存波动。其结果常表现为:产地滞销价贱与销地短缺价高的怪象并存,造成巨大的资源浪费和经济损失。

\textbf{流通冗长与效率低下:}农产品从田间到餐桌往往经历生产者→产地集散市场→多级批发商→零售商→消费者的漫长链条。每个环节的加价、仓储、物流都增加了成本和时间消耗,导致“最后一公里”价格显著高于“最初一公里”。这种低效结构不仅推高了终端价格、降低了农产品的新鲜度,也压缩了农民的实际收益。复杂的分销网络(如“兰溪花猪”案例中涉及的RFID信息采集、多环节监控等)虽然试图提升管理,但本身也增加了操作复杂性和成本。

\textbf{生鲜易腐性与高损耗风险:}生鲜蔬菜具有高度易腐性,对物流时效、温度控制(冷链)提出了严苛要求。然而,当前我国冷链物流等基础设施相对薄弱,覆盖率不足、断链现象时有发生。在缺乏精准需求预测和协同调度的传统模式下,库存积压成为常态,直接加剧了产品的物理损耗(腐烂变质)和经济损耗(被迫降价甩卖)。尤其在农产品集中上市期,损耗问题更为突出。这不仅吞噬了经营者的利润,也造成了宝贵的食物资源浪费。

\textbf{价格形成机制不透明与弹性管理缺失:}农产品价格波动剧烈,受季节、天气、替代品、节日等多重因素影响。传统模式下,批发商和零售商难以科学量化产品自身的价格弹性以及品类之间的交叉价格弹性,定价决策往往基于简单的成本加成或经验法则,无法实现动态、精细化的价格管理以平衡销量、利润与库存周转。这既可能导致利润流失(价格过低),也可能抑制消费(价格过高),更无法有效利用价格杠杆调节库存。

\textbf{质量标准化与追溯体系薄弱:}农产品,尤其是生鲜类,具有非标特性,缺乏统一的质量分级标准,导致对“优质产品”的定义模糊,定价依赖主观判断。现有追溯体系主要停留在源头追踪,难以涵盖种植过程、农业投入品使用及检验报告等关键信息。其核心影响表现为:优质农产品难以被市场识别并获得应有溢价,生产者提升品质的积极性受挫;消费者对农残、激素等问题担忧,商家需高成本建立信任;发生食品安全问题时,追溯与召回困难,品牌公信力受损。
\begin{figure}[H]
    \centering
    \includegraphics[width=1\linewidth]{痛点.pdf}
    \caption{农业供应链痛点}
    \label{fig:ex_tong}
\end{figure}
传统农业供应链管理面临多重挑战,其中信息不对称问题尤为突出。信息不对称表现为生产者与消费者之间存在多重中间环节,阻碍有效信息传递。生产者难以精准把握市场需求,消费者亦无法获知农产品真实来源与品质。此类信息传递中断不仅影响农产品销售,更导致补货决策失误,使供应链各环节库存管理陷入困境。

\subsection{问题定义}
\label{subsec:problem_definition}
较为粗略地来看,农产品在抵达消费者前通常经历三阶段流程\citep{lee2007coordination,zhao2023optimizing}:

\begin{itemize}
  \item \textbf{阶段一:} 生产者(农户)从产地采收农产品,确定面向批发商的供应价格
  \item \textbf{阶段二:} 批发商按特定采购价向生产者订货,经加价后转售给零售商
  \item \textbf{阶段三:} 零售商按特定采购价向批发商订货,经加价后销售给终端消费者
\end{itemize}

此过程中,批发商作为连接生产者与零售的关键节点,其订货量与加价率直接影响零售端的库存决策与定价策略\citep{ke2018coordinating}。因此,优化批发商订货策略对提升农业供应链整体效率至关重要。

本研究聚焦\textbf{阶段二},预期解决预测和优化的问题,旨在构建批发商数据驱动的决策优化框架\footnote{
聚焦阶段二优化的原因:一方面,中国生产者(农户)普遍缺乏高精度数据记录习惯,数据采集难度大(阶段一);另一方面,零售商分布分散且存在样本选择偏差,数据记录质量劣于批发商(阶段三)。
}。核心待解问题如下:

\begin{itemize}
  \item \textbf{问题1:} 批发环节未来进货价格与补货量的精准预测
  \item \textbf{问题2:} 基于利润最大化目标进行未来库存水平与定价水平(加价幅度)优化
\end{itemize}
通过解决以上两个问题能够解决农业供应链低效的弊端,进而促进整个农业市场的发展。
\subsection{研究贡献}
\label{subsec:research_contribution}
本文融合数据科学与农业经济学领域,面向现实问题开展研究,预期为计算机科学界、经济学界及农业供应链实践做出以下三方面贡献:

\begin{itemize}
  \item \textbf{计算机科学:} 不同于 \citet{lee2025kolmogorov}和 \citet{shen2025reduced}等学者对KAN架构效用的质疑,通过在时序数据场景验证其特征学习能力,验证了KAN架构的独特价值。
  \item \textbf{经济学:} 运用深度学习模型与价格弹性模型,实证验证农产品价格-销量间的可预测性及潜在替代关系,特别是生鲜市场的投机性,为市场干预政策提供经验证据。
  \item \textbf{农业供应链:} 构建数据驱动的预测-优化框架,实现批发商利润显著优化、损耗率降低及农业市场运行效率提升的三重改进目标。
\end{itemize}

本文后续结构安排如下:第\ref{sec:data_methodology}节阐述数据与方法论,涵盖数据来源、预测方法与优化方法;第\ref{sec:results}节呈现预测与优化结果;第\ref{sec:discussion_comparison}节讨论消融实验与非凸优化算法比较;第\ref{sec:conclusions}节总结研究结论并阐明局限性与未来研究方向。
\section{数据与方法}
\label{sec:data_methodology}
本节阐述本研究采用的数据集与方法论体系。数据来源与描述性统计分析见第\ref{subsec:data}节,预测方法详述于第\ref{subsec:forecasting_method}节,优化方法在第\ref{subsec:optimization_method}节系统说明。

\subsection{数据}
\label{subsec:data}

\subsubsection{数据来源}
\label{subsubsec:data_sources}
本研究所用数据源于中国华北地方农产品批发市场的官方交易记录,涵盖六大类生鲜蔬菜产品:花菜类、叶菜类、椒类、茄果类、食用菌类及水生块根类。数据集时间跨度为2020年7月至2023年6月,共计三年期连续观测数据。鉴于单一品类生鲜蔬菜的订货-定价策略制定复杂性,本研究将六类产品作为整体分析单元\footnote{尽管只涉及六类产品,但是本研究的预测性能十分优秀,可以预见的是更大的数据集会进一步提升模型的性能。}。通过对历史销量、批发价格、损耗率及零售价格等核心指标的联合分析,构建综合预测模型与优化策略,以实现产品组合的全局订货-定价方案优化。

\subsubsection{描述性分析}
\label{subsubsec:descriptive_analysis}
本研究聚焦生鲜蔬菜销量与成本间关联关系,重点解析不同类别生鲜蔬菜的销售表现与价格波动特征。采用折线图可视化各类蔬菜销量趋势与进货价格轨迹,并构建相关系数矩阵揭示品类间销售关联。

\begin{figure}[H]
    \centering
    \includegraphics[width=1\textwidth]{图片1.png}
    \caption{描述性分析结果。(A)销售趋势 (B)品类间销量相关系数矩阵 (C)进货价格趋势}
    \label{fig:fig1}
\end{figure}

图\ref{fig:fig1}(A)显示不同生鲜蔬菜进货价格波动轨迹。水生块根类价格显著高于其他品类且波动剧烈,表明其进货成本具有较大不确定性;而花菜类、叶菜类、椒类、茄果类与食用菌类价格相对平稳,反映供应链成本稳定性较高。

图\ref{fig:fig1}(B)呈现品类间销量的皮尔逊相关系数矩阵。花菜类、叶菜类与椒类间存在显著正相关性(p<0.05),暗示其需求模式受共同因素驱动或存在互补效应;食用菌类与水生块根类则表现出弱相关性,显示独立需求特征。

图\ref{fig:fig1}(C)展示各类蔬菜销量时序趋势,具有明显的消费习惯与供应链周期驱动的波动规律。叶菜类与椒类销量位居前列,反映强劲的市场需求;水生块根类受高价制约销量最低,符合需求规律预期。
\subsection{预测方法}
\label{subsec:forecasting_method}

\subsubsection{LSTM模型}
\label{subsubsec:lstm_model}
长短期记忆网络(LSTM)是由\citet{Hochreiter1997}提出的特殊循环神经网络结构,旨在解决传统RNN中的长期依赖问题。  
LSTM基本单元包含多个关键组件:输入门、遗忘门、输出门和细胞状态\citep{Hochreiter1997}。通过门控机制,LSTM单元动态决策各时间步需记忆的信息、遗忘的信息及输出内容\citep{Hochreiter1997}。设$x_t$为时间步$t$的输入向量,$h_{t-1}$为时间步$t-1$的隐藏状态,$C_t$为时间步$t$的细胞状态。LSTM各组件更新计算如下:
\begin{align}  
f_t &= \sigma(W_f x_t + U_f h_{t-1} + b_f) \label{eq:forget_gate}\\  
i_t &= \sigma(W_i x_t + U_i h_{t-1} + b_i) \label{eq:input_gate}\\  
\tilde{c}_t &= \tanh(W_c x_t + U_c h_{t-1} + b_c) \label{eq:candidate_cell}\\  
c_t &= f_t \odot c_{t-1} + i_t \odot \tilde{c}_t \label{eq:memory_update}\\  
o_t &= \sigma(W_o x_t + U_o h_{t-1} + b_o) \label{eq:output_gate}\\  
h_t &= o_t \odot \tanh(c_t) \label{eq:hidden_update}  
\end{align}
公式\eqref{eq:forget_gate}定义遗忘门:控制历史信息保留程度,实现长期记忆管理  
公式\eqref{eq:input_gate}定义输入门:决定新信息与当前记忆状态的相关性  
公式\eqref{eq:candidate_cell}计算基于当前输入的候选记忆细胞  
公式\eqref{eq:memory_update}通过遗忘门与输入门输出组合更新细胞状态  
公式\eqref{eq:output_gate}定义输出门:调控传递至下一层或时间步的信息  
公式\eqref{eq:hidden_update}更新后续时间步的隐藏状态  
LSTM模型整体结构如图\ref{fig:fig2}所示。

\begin{figure}[H]
        \centering
        \includegraphics[width=1\textwidth]{图片2.png}
        \caption{LSTM模型架构}
        \label{fig:fig2}
\end{figure}

\subsubsection{KAN-LSTM模型}
\label{subsubsec:kan_lstm_model}

柯尔莫哥洛夫-阿诺德网络(KAN)是基于柯尔莫哥洛夫-阿诺德表示定理的神经网络架构。其核心思想是将任意多元连续函数表示为单变量函数的叠加与复合\citep{Ismailov2008,Genet2024}。该网络架构的数学推导详见附录A。为融合KAN与LSTM优势,构建KAN-LSTM模型如下:

\paragraph{KAN层输入变换}  
KAN层通过单变量映射$\phi_{q,p}$转换输入$x_t$:  
\begin{equation}
\hat{x}_{t,q} = \sum_{p=1}^{n} \phi_{q,p} \left( W_{x,p} x_{t,p} + b_{x,p} \right)
\end{equation}

\paragraph{RKAN状态更新}  
递归柯尔莫哥洛夫-阿诺德网络(RKAN)通过递归应用KAN层变换管理时间序列依赖关系:  
\begin{equation}
h_t = \sum_{q=1}^{2n+1} \Phi_q(\tilde{x}_{t,q})
\end{equation}
其中$\Phi_q$为KAN层学习的单变量映射函数。

\paragraph{LSTM门控机制融合}  
为增强模型捕捉长短期依赖能力,将LSTM门控机制融入RKAN隐藏状态更新过程:  
\begin{align}
f_t &= \sigma(W_{f,h}h_{t-1} + W_{f,x}x_t + b_f) \\
i_t &= \sigma(W_{i,h}h_{t-1} + W_{i,x}x_t + b_i) \\
\tilde{c}_t &= \tanh(W_{c,h}h_{t-1} + W_{c,x}x_t + b_c) \\
c_t &= f_t \odot c_{t-1} + i_t \odot \tilde{c}_t \\
o_t &= \sigma(W_{o,h}h_{t-1} + W_{o,x}x_t + b_o) \\
h_t &= o_t \odot \tanh(c_t)
\end{align}
KAN-LSTM最终架构如图\ref{fig:fig3}所示:

\begin{figure}[H]
        \centering
        \includegraphics[width=1\textwidth]{图片3.png}
        \caption{KAN-LSTM模型架构}
        \label{fig:fig3}
\end{figure}

\subsubsection{模型评价指标}
\label{subsubsec:model_evaluation_metrics}
为全面评估预测值与实际观测值的偏离程度,本研究选用三种常用评价指标:决定系数($R^2$)、均方根误差(RMSE)和平均绝对百分比误差(MAPE)。其定义如下:

\begin{equation}
R^2 = 1 - \frac{\sum_{i=1}^{n} (y_i - \hat{y}_i)^2}{\sum_{i=1}^{n} (y_i - \bar{y})^2},
\end{equation}

\begin{equation}
RMSE = \sqrt{\frac{1}{n} \sum_{i=1}^{n} (y_i - \hat{y}_i)^2},
\end{equation}

\begin{equation}
MAPE = \frac{1}{n} \sum_{i=1}^{n} \left| \frac{y_i - \hat{y}_i}{y_i} \right| \times 100,
\end{equation}
其中$y_i$为实际观测值,$\hat{y}_i$为模型预测值,$\bar{y}$为实际值样本均值,$n$为样本量。上述指标将用于评估模型在训练集与测试集上的性能,验证所提方法的准确性与可靠性。
\subsection{优化方法}
\label{subsec:optimization_method}
订货策略优化需综合考虑定价、需求预测与库存管理,实现利润最大化目标。本研究通过三部分构建优化模型:成本加成定价模型、需求回归模型及非线性规划模型构建。

\subsubsection{成本加成定价模型}
\label{subsubsec:cost_plus_pricing}
成本加成定价是订货策略的基础,其核心通过在进货成本上加成确定产品销售价格。定价公式为:
\begin{equation}
p_i = c_i  (1 + m_i),
\label{eq:cs1}
\end{equation}
其中$p_i$表示产品销售价格,$c_i$表示产品进货成本,$m_i$表示产品加价率。

合理的加价率$m_i$是后续规划模型中的关键决策变量。确定最优加价率需结合需求回归模型与规划目标进行综合优化。

\subsubsection{需求回归模型}
\label{subsubsec:demand_regression_model}
产品需求$D_i$不仅取决于自身价格$p_i$,更受相关产品价格$p_j$与销量$D_j$的非线性影响。为捕捉价格弹性与交叉弹性效应,基于弹性理论构建多元需求回归模型,其对数形式为:

\begin{equation}
\ln(D_i) = \alpha - \beta \ln(p_i) + \sum_j \gamma_j \ln(p_j) + \sum_j \delta_j \ln(D_j),
\label{eq:cs2}
\end{equation}

式中$\alpha$为规模参数,$\beta$为自身价格弹性,$\gamma_j$、$\delta_j$分别为交叉价格弹性与交叉销量弹性系数。

结合成本加成定价公式,需求函数转换为:

\begin{equation}
D_i = e^{\alpha} [c_i(1 + m_i)]^{-\beta} \prod_j p_j^{\gamma_j} \prod_j D_j^{\delta_j}.
\label{eq:cs2}
\end{equation}

通过估计参数$\alpha$、$\beta$、$\gamma_j$和$\delta_j$,可量化自身价格效应、交叉价格效应及交叉销量关系对需求的影响,为定价与订货决策提供依据。

\subsubsection{非线性规划模型构建}
\label{subsubsec:nonlinear_programming_model}
基于成本加成定价与需求回归模型,以进货量$Q_i$和加价率$m_i$为决策变量,构建考虑库存成本的利润最大化非线性规划模型:

\paragraph{目标函数}
店铺总利润由销售利润与库存成本构成:
\begin{equation}
\max_{Q, m} \quad \text{Profit} = \sum_i [(p_i - c_i)  D_i - h_i  (Q_i - D_i)],
\end{equation}
其中$(p_i - c_i)  D_i$为产品销售利润,$h_i  (Q_i - D_i)$为库存持有成本,$h_i$为第$i$种产品损耗率,$Q_i$为产品进货量\footnote{在本文使用的数据集中,农贸市场并未形成日度记录损耗率的制度,因此本文中的损耗率均为不随时间变化的固定值,详细见https://github.com/Zhanli-Li/Digital-Agricultural-Economics}。

\paragraph{约束条件1:需求满足约束}
订货量至少需满足需求,确保库存非负:
\begin{equation}
Q_i \geq e^{\alpha_i} \cdot \left[ c_i \cdot (1 + m_i) \right]^{-\beta_i} \prod_j p_j^{\gamma_{ij}} \prod_j D_j^{\delta_{ij}}, \quad \forall i.
\end{equation}

\paragraph{约束条件2:加价率范围约束}
加价率需处于合理区间:
\begin{equation}
m_{\text{min}} \leq m_i \leq m_{\text{max}}, \quad \forall i.
\end{equation}


整合目标函数与约束条件(含式\eqref{eq:cs1}与式\eqref{eq:cs2}),得完整优化模型:

\textbf{目标函数:}
\begin{equation}
\max_{Q, m} \quad \text{Profit} = \sum_i \left[ (p_i - c_i) \cdot D_i - h_i \cdot (Q_i - D_i) \right].
\end{equation}

\textbf{约束条件:}
\begin{equation}
\begin{cases}
Q_i &\geq e^{\alpha_i} \cdot \left[ c_i \cdot (1 + m_i) \right]^{-\beta_i} \prod_j p_j^{\gamma_{ij}} \prod_j D_j^{\delta_{ij}}, \quad \forall i \\
m_{\min} &\leq m_i \leq m_{\max}, \quad \forall i \\
D_i &= e^{\alpha_i} \cdot \left[ c_i \cdot (1 + m_i) \right]^{-\beta_i} \prod_j p_j^{\gamma_{ij}} \prod_j D_j^{\delta_{ij}} \\
p_i &= c_i \cdot (1 + m_i), \quad \forall i
\end{cases}
\end{equation}
\section{结果分析}
\label{sec:results}
本节展示KAN-LSTM模型预测性能与订货策略优化的实证结果。第\ref{subsec:prediction_results}节报告预测模型结果,第\ref{subsec:ordering_strategy}节阐述订货策略优化结果。

\subsection{KAN-LSTM模型预测结果}
\label{subsec:prediction_results}
本研究基于花菜类等六类生鲜蔬菜历史销售数据构建KAN-LSTM模型。模型训练超参数设置如表\ref{tab:hyperparameters}所示,数据集按7:3比例划分为训练集与测试集。

\begin{table}[H]
\centering
\caption{超参数配置}
\label{tab:hyperparameters}
\begin{tabular}{lcc}
\toprule
\textbf{超参数} & \textbf{取值} \\
\midrule
隐藏层数量 & 2 \\
单隐藏层单元数 & 6 \\
全连接层神经元数 & 7 \\
激活函数 & ReLU \\
优化器 & Adam \\
学习率 & 0.001 - 0.005 \\
损失函数 & 均方误差(MSE) \\
训练轮次 & 400 \\
\bottomrule
\end{tabular}
\end{table}

基于上述参数配置进行模型训练,记录损失函数值变化过程。通过损失函数曲线收敛趋势分析,证实KAN-LSTM模型在各类数据集上均达到稳定收敛状态。

\begin{figure}[H]
    \centering
    \includegraphics[width=1\textwidth]{图片4.png}
    \caption{预测性能评估。(A)销量数据预测效果 (B)进货价格预测效果}
    \label{fig:fig4}
\end{figure}

图\ref{fig:fig4}(A)(B)显示:KAN-LSTM模型在销量与进货价格预测中均表现优异。$R^2$值表明花菜类、叶菜类等六类产品的预测值与实际值高度吻合,模型整体拟合优度优越。RMSE与MAPE指标在多数品类处于较低水平,进一步验证模型捕捉复杂时间序列模式及非线性关系的能力。但叶菜类等品类RMSE值偏高,反映其数据波动性或市场不确定性较强。综合而言,KAN-LSTM在农产品销量与价格预测中展现出色精度。

\subsection{订货策略优化结果}
\label{subsec:ordering_strategy}

\subsubsection{需求回归模型参数估计}
\label{subsubsec:demand_regression_results}
采用多元逐步回归法求解需求模型,通过回归系数及其显著性水平(P值)识别影响需求变动的关键解释变量。最终回归模型参数估计及显著性结果见图\ref{fig:fig5}与表\ref{tab:param_estimates}。值得关注的是,大多数农产品价格与需求呈现正相关,这意味着农业供应链中,生鲜产品极大可能为一种投机产品。

如图\ref{fig:fig5}所示,六类蔬菜需求回归模型拟合良好:调整后$R^2$值介于0.545-0.839,F统计量均显著(p<0.05),表明模型整体显著且能有效解释需求变动。基于预测结果构建非线性规划模型如下:

\begin{equation}
\max_{Q,m} \quad \text{Profit} = \sum_i \left[ (p_i - c_i) \cdot D_i - h_i \cdot (Q_i - D_i) \right]
\end{equation}
\text{约束条件:}
\begin{equation}
\begin{cases}
Q_i \geq e^{\alpha_i} \cdot \left[ c_i \cdot (1 + m_i) \right]^{-\beta_i} \prod_j p_j^{\gamma_{ij}} \prod_j D_j^{\delta_{ij}}, & \forall i \\
m_{\min} \leq m_i \leq m_{\max}, & \forall i \\
D_C = e^{2.1886} \cdot p_C^{-0.9857} \cdot p_S^{0.5908} \cdot D_M^{0.4195} \cdot D_L^{0.4908}, & \text{花菜类需求} \\
D_L = e^{3.1204} \cdot p_L^{-0.6574} \cdot p_C^{0.3061} \cdot D_P^{0.2717} \cdot D_M^{0.3482} \cdot D_C^{0.2072}, & \text{叶菜类需求} \\
D_P = e^{3.2775} \cdot p_P^{-0.5992} \cdot p_M^{0.2702} \cdot D_M^{0.5095} \cdot D_S^{0.1585} \cdot D_C^{0.1067}, & \text{椒类需求} \\
D_S = e^{-2.9685} \cdot p_C^{0.5727} \cdot D_P^{0.6752} \cdot D_M^{0.4111}, & \text{茄果类需求} \\
D_M = e^{-0.4126} \cdot p_M^{-0.3457} \cdot p_P^{0.5006} \cdot D_P^{0.7163} \cdot D_L^{0.3453}, & \text{食用菌类需求} \\
D_A = e^{-0.9428} \cdot p_L^{-1.5807} \cdot p_S^{0.7967} \cdot D_P^{1.0462}, & \text{水生块根类需求} \\
p_i = c_i \cdot (1 + m_i), & \forall i
\end{cases}
\end{equation}

其中$D_C$、$D_L$、$D_P$、$D_S$、$D_M$、$D_A$分别表示花菜类、叶菜类、椒类、茄果类、食用菌类及水生块根类销量;$P_C$、$P_L$、$P_P$、$P_S$、$P_M$、$P_A$表示相应品类批发价格(含加价率)。

\begin{figure}[H]
    \centering
    \includegraphics[width=1\textwidth]{图片5.png}
    \caption{需求回归模型估计结果}
    \label{fig:fig5}
\end{figure}

\begin{table}[H]
    \centering
    \caption{参数估计与显著性检验结果}
    \begin{tabular}{cccccc}
    \toprule
    \textbf{模型I} & \textbf{模型II} & \textbf{模型III} & \textbf{模型IV} & \textbf{模型V} & \textbf{模型VI} \\
    \midrule
    \textbf{常数项} & \textbf{常数项} & \textbf{常数项} & \textbf{常数项} & \textbf{常数项} & \textbf{常数项} \\
    (2.1886**) & (3.1204***) & (3.2775***) & (-2.6385***) & (-0.4126) & (-0.9428) \\
    $p_C$ & $p_L$ & $p_P$ & $p_C$ & $p_M$ & $p_L$ \\
    (-0.9857***) & (-0.6574***) & (-0.5992***) & (0.5727***) & (-0.3457***) & (-1.5807***) \\
    $p_S$ & $p_C$ & $p_M$ & $p_P$ & $p_P$ & $p_S$ \\
    (0.5908**) & (0.3061**) & (0.2702**) & (0.6752***) & (0.5006***) & (0.7967**) \\
    $D_M$ & $D_P$ & $D_M$ & $D_M$ & $D_P$ & $D_P$ \\
    (0.4195***) & (0.2717**) & (0.5095***) & (0.4111**) & (0.7163***) & (1.0462***) \\
    $D_L$ & $D_M$ & $D_S$ & \ & $D_L$ & \ \\
    (0.4908***) & (0.3482***) & (0.1585***) & \ & (0.3453***) & \ \\
    \ & $D_C$ & $D_C$ & \ & \ & \ \\
    \ & (0.2072***) & (0.1067**) & \ & \ & \ \\
    \bottomrule
    \end{tabular}
    \label{tab:param_estimates}
\end{table}
\begin{tablenotes}
\footnotesize
\item 注:模型I-VI分别对应$D_C$、$D_L$、$D_P$、$D_S$、$D_M$、$D_A$,括号内为系数估计值与显著性水平。***、**、*分别表示1\%、5\%、10\%显著性水平。
\end{tablenotes}

\subsubsection{未来15日订货决策求解}
\label{subsubsec:ordering_decision_15_days}

\textbf{定理1:} 在$\beta_i \leq -1$或$\beta_i \geq 0$时,本文提出的批发市场供应链管理与订货策略优化问题为凸优化问题。

具体而言,若各类产品需求函数$D_i$满足:
\begin{equation}
D_i = e^{\alpha_i} \left[ c_i (1 + m_i) \right]^{-\beta_i} \prod_j p_j^{\gamma_{ij}} \prod_j D_j^{\delta_{ij}},
\end{equation}
且$\beta_i \leq -1$或$\beta_i \geq 0$,则目标函数与约束条件定义的可行域为凸集。此时优化问题具有凸性,可采用凸优化算法获取全局最优解。定理证明见附录B。

基于表\ref{tab:param_estimates}需求函数及定理1分析:模型I($\beta_i$=-0.9857)和模型II($\beta_i$=-0.6574)满足凹函数条件;模型III($\beta_i$=-0.5992)与模型IV($\beta_i$=-0.5727)仍保持非凸性;模型V($\beta_i$=-0.3457)亦为非凸;模型VI因逐步回归剔除价格弹性变量而成为凸函数。整体而言,订货决策模型呈凹性结构。

需求函数凸凹性差异表明优化问题可能存在非凸性,增加全局最优解搜索难度。因此采用粒子群优化(PSO)、遗传算法(GA)等全局搜索算法提升解的可靠性与有效性。

基于需求预测模型与目标函数,本研究采用模拟退火算法\citep{Tavares2011}求解未来15日订货量与加价率。优化过程引入约束条件保证解可行性,最终实现总利润最大化(图\ref{fig:fig6})。

\begin{figure}[H]
    \centering
    \includegraphics[width=1\textwidth]{图片6.png}
    \caption{未来15日超市运营决策结果。(A)定价加价率决策 (B)订货量决策 (C)农贸市场日利润}
    \label{fig:fig6}
\end{figure}

优化结果显示:加价率范围0.38-0.50,花菜类与水生块根类加价率较低,茄果类较高;日采购量呈波动特征,叶菜类采购量最高,水生块根类最低;超市日利润区间764.64-1123.22元,整体呈波动上升趋势,证实优化策略有效性。
\section{讨论与比较}
\label{sec:discussion_comparison}
本节探讨KAN-LSTM模型在农产品销量与价格预测中的性能表现,内容分为两部分:第\ref{subsec:ablation_study}节分析时序长期记忆的消融实验,第\ref{subsec:discussion_ordering_strategy}节讨论订货策略优化结果。

\subsection{时序长期记忆消融实验}
\label{subsec:ablation_study}

为验证KAN-LSTM模型的长期记忆能力,本研究将数据集划分为1年、2年、3年三种预测窗口,选取LSTM与xLSTM作为基准模型,评估不同预测跨度下模型的泛化性能。

\begin{table}[H]
  \centering
  \caption{不同时间窗口销量预测的KAN-LSTM消融实验}
  \label{tab:abl_sales}
  \resizebox{\textwidth}{!}{ 
  \begin{tabular}{lcccccccccc}
    \toprule
模型 &  & \multicolumn{3}{c}{\textbf{KAN-LSTM}} & \multicolumn{3}{c}{\textbf{xLSTM\citep{Beck2024}}} & \multicolumn{3}{c}{\textbf{LSTM\citep{Hochreiter1997}}} \\
\cmidrule(lr){3-5} \cmidrule(lr){6-8} \cmidrule(lr){9-11} 
指标 & 窗口 & \textbf{R2} & \textbf{RMSE} & \textbf{MAPE} & \textbf{R2} & \textbf{RMSE} & \textbf{MAPE} & \textbf{R2} & \textbf{RMSE} & \textbf{MAPE} \\
\midrule
    \multirow{3}{*}{花菜类} 
      & 365天 & \color{red}0.9769 & \color{red}3.5720 & \color{red}5.2679 & 0.9569 & 2.1479 & 7.3288 & 0.9320 & 2.6979 & 7.7506 \\
      & 730天 & \color{red}0.9710 & \color{red}3.1740 & \color{red}3.5875 & 0.9651 & 3.4821 & 2.7741 & 0.9346 & 4.7667 & 2.8250 \\
      & 1095天 & \textbf{\color{red}0.9903} & \textbf{\color{red}2.4215} & \textbf{\color{red}2.1889} & 0.9745 & 3.9370 & 4.2280 & 0.9335 & 6.3585 & 6.7969 \\
    \multirow{3}{*}{叶菜类} 
      & 365天 & \color{red}0.9018 & \textbf{\color{red}6.3052} & \color{red}1.9257 & 0.9421 & 11.2439 & 4.1950 & 0.9495 & 10.4983 & 3.1545 \\
      & 730天 & \color{red}0.9188 & \color{red}14.1301 & \color{red}1.6266 & 0.9285 & 21.3790 & 2.1547 & 0.9301 & 21.1357 & 1.7232 \\
      & 1095天 & \textbf{\color{red}0.9307} & \color{red}34.6191 & \textbf{\color{red}1.5977} & 0.8838 & 37.4981 & 2.6139 & 0.8695 & 39.7413 & 2.9743 \\
    \multirow{3}{*}{椒类} 
      & 365天 & \color{red}0.9673 & \color{red}7.1016 & \color{red}3.3214 & 0.9298 & 8.9425 & 2.6200 & 0.9271 & 9.1136 & 4.2099 \\
      & 730天 & \color{red}0.9508 & \color{red}12.6713 & \color{red}3.0132 & 0.9411 & 13.8579 & 4.2194 & 0.9330 & 14.7786 & 4.0132 \\
      & 1095天 & \textbf{\color{red}0.9864} & \textbf{\color{red}6.9606} & \textbf{\color{red}1.6975} & 0.9627 & 11.5436 & 6.6992 & 0.9552 & 12.6497 & 5.7077 \\
    \multirow{3}{*}{茄果类} 
      & 365天 & \color{red}0.9873 & \color{red}1.1469 & \textbf{\color{red}3.1911} & 0.9747 & 1.4785 & 3.8310 & 0.9559 & 1.9511 & 5.6651 \\
      & 730天 & \color{red}0.9865 & \color{red}1.3101 & \color{red}5.6574 & 0.9724 & 1.8717 & 8.7564 & 0.9599 & 2.2543 & 7.1288 \\
      & 1095天 & \textbf{\color{red}0.9912} & \textbf{\color{red}1.0600} & \color{red}5.9722 & 0.9536 & 2.4296 & 14.3475 & 0.9450 & 2.6457 & 26.3232 \\
    \multirow{3}{*}{食用菌类} 
      & 365天 & \color{red}0.9834 & \color{red}5.4062 & \color{red}3.1830 & 0.9660 & 4.8710 & 4.2744 & 0.9599 & 5.2907 & 3.8349 \\
      & 730天 & \color{red}0.9652 & \color{red}9.7093 & \color{red}2.6739 & 0.9632 & 9.9824 & 3.9105 & 0.9409 & 12.6473 & 2.7737 \\
      & 1095天 & \textbf{\color{red}0.9935} & \textbf{\color{red}2.1291} & \textbf{\color{red}1.0923} & 0.9315 & 5.2424 & 3.5220 & 0.9188 & 5.2639 & 3.1279 \\
    \multirow{3}{*}{水生块根类} 
      & 365天 & \color{red}0.9846 & \color{red}5.0668 & \color{red}5.5761 & 0.9470 & 1.9774 & 10.2619 & 0.9345 & 2.1975 & 13.3699 \\
      & 730天 & \color{red}0.9521 & \color{red}7.1925 & \color{red}5.6544 & 0.9202 & 9.2878 & 7.3120 & 0.9364 & 8.2901 & 4.3246 \\
      & 1095天 & \textbf{\color{red}0.9872} & \textbf{\color{red}4.0699} & \textbf{\color{red}5.5044} & 0.9750 & 5.6879 & 9.5540 & 0.9511 & 7.9602 & 16.6544 \\
    \bottomrule
  \end{tabular}
  }
\end{table}

\begin{table}[H]
  \centering
  \caption{不同时间窗口进货价格预测的KAN-LSTM消融实验}
  \label{tab:abl_prices}
  \resizebox{\textwidth}{!}{
  \begin{tabular}{lcccccccccc}
  \toprule
模型 & 窗口 & \multicolumn{3}{c}{\textbf{KAN-LSTM}} & \multicolumn{3}{c}{\textbf{xLSTM\citep{Beck2024}}} & \multicolumn{3}{c}{\textbf{LSTM\citep{Hochreiter1997}}} \\
\cmidrule(lr){3-5} \cmidrule(lr){6-8} \cmidrule(lr){9-11} 
指标 &  & \textbf{R2} & \textbf{RMSE} & \textbf{MAPE} & \textbf{R2} & \textbf{RMSE} & \textbf{MAPE} & \textbf{R2} & \textbf{RMSE} & \textbf{MAPE} \\
\midrule
    \multirow{3}{*}{花菜类} 
      & 365天 & \color{red}0.9604 & \color{red}0.1620 & \color{red}1.9586 & 0.9319 & 0.2125 & 1.3935 & 0.9257 & 0.2218 & 0.8013 \\
      & 730天 & \color{red}0.9565 & \color{red}0.2015 & \color{red}1.5385 & 0.9319 & 0.2519 & 1.8398 & 0.9446 & 0.2273 & 1.7983 \\
      & 1095天 & \textbf{\color{red}0.9752} & \textbf{\color{red}0.1378} & \textbf{\color{red}1.0163} & 0.9515 & 0.1930 & 1.1940 & 0.9203 & 0.2473 & 1.9973 \\
    \multirow{3}{*}{叶菜类} 
      & 365天 & \color{red}0.9830 & \textbf{\color{red}0.0639} & \color{red}1.6084 & 0.9451 & 0.1794 & 1.9208 & 0.9044 & 0.2366 & 3.3553 \\
      & 730天 & \color{red}0.9852 & \color{red}0.0942 & \color{red}1.7392 & 0.9629 & 0.1494 & 2.5803 & 0.9221 & 0.2164 & 3.4119 \\
      & 1095天 & \textbf{\color{red}0.9880} & \color{red}0.0977 & \textbf{\color{red}1.0796} & 0.9580 & 0.1826 & 3.8172 & 0.9144 & 0.2608 & 6.4929 \\
    \multirow{3}{*}{椒类} 
      & 365天 & \color{red}0.9501 & \color{red}0.1951 & \color{red}0.5621 & 0.9501 & 0.1063 & 0.6362 & 0.9036 & 0.1478 & 0.7433 \\
      & 730天 & \color{red}0.9511 & \color{red}0.1267 & \color{red}0.9111 & 0.9334 & 0.1479 & 1.2837 & 0.9345 & 0.1466 & 1.2799 \\
      & 1095天 & \textbf{\color{red}0.9517} & \textbf{\color{red}0.1161} & \textbf{\color{red}0.4079} & 0.9464 & 0.1224 & 1.5457 & 0.9368 & 0.1329 & 1.7060 \\
    \multirow{3}{*}{茄果类} 
      & 365天 & \color{red}0.9715 & \textbf{\color{red}0.1393} & \color{red}0.6615 & 0.9445 & 0.1943 & 1.1448 & 0.9459 & 0.1918 & 1.2810 \\
      & 730天 & \color{red}0.9548 & \color{red}0.1565 & \color{red}0.8651 & 0.9627 & 0.1422 & 1.2729 & 0.9214 & 0.2065 & 2.5518 \\
      & 1095天 & \textbf{\color{red}0.9756} & \color{red}0.1710 & \textbf{\color{red}0.5420} & 0.9577 & 0.2251 & 0.9828 & 0.9428 & 0.2618 & 1.2196 \\
    \multirow{3}{*}{食用菌类} 
      & 365天 & \color{red}0.9801 & \color{red}0.1311 & \color{red}1.6576 & 0.9564 & 0.1687 & 1.2552 & 0.9073 & 0.2460 & 1.6336 \\
      & 730天 & \color{red}0.9617 & \color{red}0.1843 & \color{red}1.3138 & 0.9331 & 0.2435 & 1.6365 & 0.9055 & 0.2893 & 2.0250 \\
      & 1095天 & \textbf{\color{red}0.9806} & \textbf{\color{red}0.1291} & \textbf{\color{red}1.0923} & 0.9315 & 0.2424 & 2.5220 & 0.9188 & 0.2639 & 3.1279 \\
    \multirow{3}{*}{水生块根类} 
      & 365天 & \color{red}0.9704 & \color{red}0.5750 & \color{red}1.8830 & 0.9524 & 0.7286 & 2.9308 & 0.9006 & 1.0527 & 4.9100 \\
      & 730天 & \color{red}0.9829 & \color{red}0.3734 & \color{red}1.8555 & 0.9540 & 0.6124 & 4.5322 & 0.9122 & 0.8461 & 5.4465 \\
      & 1095天 & \textbf{\color{red}0.9941} & \textbf{\color{red}0.1884} & \textbf{\color{red}1.4035} & 0.9608 & 0.4843 & 1.9060 & 0.9276 & 0.6588 & 4.0008 \\
    \bottomrule
  \end{tabular}
  }
\end{table}

表\ref{tab:abl_sales}与\ref{tab:abl_prices}显示:KAN-LSTM在所有时间窗口(365/730/1095天)均优于LSTM和xLSTM。花菜类销量预测中,KAN-LSTM的$R^2$值达0.9903(1095天窗口),显著高于对比模型;其RMSE与MAPE值也保持最低,验证预测准确性优势。

椒类销量预测在365天窗口下,KAN-LSTM的RMSE(7.1016)和MAPE(3.3214)均优于基准模型。特别在1095天长期预测中,KAN-LSTM展现更强的长期依赖捕捉能力:水生块根类进货价格预测$R^2$达0.9941,显著高于LSTM(0.9276)和xLSTM(0.9608),RMSE(0.1884)与MAPE(1.4035)改进幅度超30%。

结果表明:KAN-LSTM在长期时序预测中具有显著优势,能有效建模历史动态模式。1095天窗口验证集预测结果见图\ref{fig:fig7}-\ref{fig:fig8},其他窗口结果见附录C。

\begin{figure}[H]
    \centering
    \includegraphics[width=1\textwidth]{图片7.png}
    \caption{1095天窗口销量预测结果。(A)-(F)依次为花菜类、叶菜类、椒类、茄果类、食用菌类、水生块根类}
    \label{fig:fig7}
\end{figure}

\begin{figure}[H]
    \centering
    \includegraphics[width=1\textwidth]{图片8.png}
    \caption{1095天窗口进货价格预测结果。(A)-(F)品类顺序同图\ref{fig:fig7}}
    \label{fig:fig8}
\end{figure}

\subsection{订货策略优化讨论}
\label{subsec:discussion_ordering_strategy}

为分析模拟退火(SA)算法在求解非凸优化问题中的优势,选取粒子群优化(PSO)、遗传算法(GA)与禁忌搜索(TS)进行对比\citep{Kennedy1995,Alhijawi2024,Niroumandrad2024}。通过优化未来15日补货量与加价率,评估各算法求解多约束非线性规划的性能差异(图\ref{fig:fig9})。

结果显示:模拟退火(SA)、遗传算法(GA)与禁忌搜索(TS)优化结果相近,但GA与TS单次优化耗时高达3小时;粒子群优化(PSO)虽可在3分钟内完成,但解质量较差;SA算法兼顾优化效率与质量,单周期优化仅需约1分钟。这表明SA在时效敏感的实际决策场景中具有显著优势,特别适合需频繁调整的供应链动态优化问题。

\begin{figure}[H]
    \centering
    \includegraphics[width=1\textwidth]{图片9.png}
    \caption{优化算法性能比较。(A)SA算法 (B)GA算法 (C)PSO算法 (D)TS算法}
    \label{fig:fig9}
\end{figure}
\section{研究结论}
\label{sec:conclusions}

本文针对生鲜蔬菜产品的库存与定价策略优化问题,提出基于需求预测与优化决策的解决框架。通过多元逐步回归模型估计产品价格弹性与交叉弹性,建立融合产品成本、需求-价格关系及库存持有成本的需求预测模型,并采用KAN-LSTM深度学习模型实现需求时序预测。该模型展现优异的泛化能力,可有效捕捉复杂时序动态特征。

优化阶段运用模拟退火算法确定未来15日库存与加价策略,在保证解可行性与计算效率同时实现利润最大化。对比结果表明:模拟退火算法的计算效率显著优于其他算法,且优化结果稳定可靠。这些发现验证了框架有效性,为农产品市场库存定价决策提供科学依据,拓展了需求预测与优化算法在现实场景的应用潜力。

本研究存在一定局限性:首先未考虑损耗率($h$)随时间变化的动态特性;其次蔬菜补货与定价行为与天气、温度等自然因素存在关联,尽管历史数据已隐含此类信息,但显式引入自然变量能否进一步提升结果精度仍需深入探讨。同时本文的预测模型为深度学习模型,缺少可解释性,如何使用参数较少的计量时序模型研究该问题也是值得探究的。
\newpage
\section*{数据可用性声明}
本研究数据通过\url{https://github.com/Zhanli-Li/Digital-Agricultural-Economics}公开访问,使用权限遵循Apache-2.0开源许可协议。
\section*{AI辅助声明}
Copilot (Education Benefits Version) 用于辅助编写Python代码,我已检查代码并确保其正确性。ChatGPT 和 DeepSeek 用于辅助润色文章和检查语病,我已检查内容并确保其正确性。代码或内容中的任何错误由我本人负责。
\newpage
\nocite{*}
\printbibliography[heading=bibintoc, title=\ebibname]
\newpage
\appendix
%\appendixpage
\addappheadtotoc
\section{附录:KAN网络架构}
\label{sec:appendix_kan}

柯尔莫哥洛夫-阿诺德表示定理表明:任意多元连续函数$f(x_1, x_2, \ldots, x_n)$可表示为一元函数的叠加与复合\citep{Braun2009},即:
\begin{equation}
f(x_1, \ldots, x_n) = \sum_{q=1}^{2n+1} \Phi_q \left( \sum_{p=1}^n \phi_{q,p}(x_p) \right),
\end{equation}
其中$\phi_{q,p} : [0,1] \rightarrow \mathbb{R}$为一元映射函数,$\Phi_q : \mathbb{R} \rightarrow \mathbb{R}$为另一组一元映射函数。

\begin{figure}[H]
        \centering
        \includegraphics[width=1\textwidth]{图片14.png}
        \caption{KAN架构运行机理}
        \label{fig:fig14}
\end{figure}
如图\ref{fig:fig14}所示,KAN网络架构基于该理论,在神经网络连接处学习激活函数而非使用传统固定激活函数(如ReLU、Sigmoid)。KAN网络中每层计算式为:
\begin{equation}
x_{l+1,j} = \sum_{i=1}^{n_l} \phi_{l,j,i}(x_{l,i}),
\end{equation}
可表示为:
\begin{equation}
x_{l+1} = \Phi_l x_l,
\end{equation}
其中$\Phi_l$表示KAN网络的函数矩阵。KAN网络整体计算表示为:
\begin{equation}
KAN(x) = (\Phi_L \circ \Phi_{L-1} \circ \cdots \circ \Phi_1 \circ \Phi_0) x.
\end{equation}
该架构赋予KAN更强的非线性建模能力与优越的函数逼近特性。

\section{附录:定理1证明}
\label{sec:appendix_proof}
为证明本文定理,需引入以下引理:
\label{sec:proof}
\begin{lemma}[复合函数凹凸性传递规则]
        若$f(x)$为凹函数且$g(x)$单调递增,则$f(g(x))$仍为凹函数;若$f(x)$为凸函数且$g(x)$单调递增,则$f(g(x))$仍为凸函数。
    \end{lemma}
    
    \begin{lemma}[单调函数与凹凸函数的乘积规则]
        设$f(x)$为凸函数,$g(x)$为单调递增线性函数,则乘积函数$h(x) = f(x) g(x)$为凸函数;若$f(x)$为凹函数,$g(x)$为单调递增线性函数,则$h(x)$为凹函数。
    \end{lemma}
    
    \begin{lemma}[线性函数的凹凸性传递]
        给定函数$g(x)$与线性函数$h(x) = ax + b$:若$g(x)$为凹函数,则$f(x) = g(x) - h(x)$仍为凹函数;若$g(x)$为凸函数,则$f(x) = g(x) - h(x)$仍为凸函数。
    \end{lemma}

首先解析本文目标函数与约束条件。目标函数包含利润项与成本项两部分。根据引理1,仅需分别论证两部分函数的凹凸性。

\textbf{利润函数}
利润定义为:
\begin{equation}
f_1(Q, m) = \sum_i c_i m_i D_i
\end{equation}
其中$c_i m_i$为线性递增函数,不改变凹凸性。因需求函数$D_i$与$m_i$呈非线性关系,由引理2,$f_1(Q, m)$的凹凸性取决于需求函数$D_i$的凹凸性。

\textbf{成本函数}
成本定义为:
\begin{equation}
f_2(Q, m) = \sum_i h_i (D_i - Q_i)
\end{equation}
此处$Q_i$为线性变量,不改变凹凸性。因$D_i$与$m_i$的非线性关系,由引理3,成本函数$f_2(Q, m)$的凹凸性与$D_i$保持一致。

\textbf{约束条件}
库存约束$Q_i \geq D_i$:若$D_i$为凸函数,则$Q_i \geq D_i$定义的集合为凸集;反之若$D_i$为非凸函数,则该约束定义非凸集。
加价率约束$m_{\min} \leq m_i \leq m_{\max}$为线性约束,其定义的集合为凸集。

综上,目标函数与约束条件的凹凸性完全取决于$D_i$的凹凸性。

\textbf{需求函数凹凸性分析}
需求函数$D_i$形式为:
\begin{equation}
D_i = e^{\alpha_i} \left[ c_i (1 + m_i) \right]^{-\beta_i} \prod_j p_j^{\gamma_{ij}} \prod_j D_j^{\delta_{ij}}
\end{equation}
因$p_j^{\gamma_{ij}}$不直接依赖$m_i$,且$D_j^{\delta_{ij}}$与$m_i$关系需递归求解,现仅考虑$D_i$对$m_i$的二阶特性。$D_i$对$m_i$的一阶与二阶导数为:
\begin{equation}
\frac{\partial D_i}{\partial m_i} = -\beta_i e^{\alpha_i} \left[ c_i (1 + m_i) \right]^{-\beta_i - 1} c_i \prod_j p_j^{\gamma_{ij}} \prod_j D_j^{\delta_{ij}}
\end{equation}
\begin{equation}
\frac{\partial^2 D_i}{\partial m_i^2} = \beta_i (\beta_i + 1) e^{\alpha_i} \left[ c_i (1 + m_i) \right]^{-\beta_i - 2} c_i^2 \prod_j p_j^{\gamma_{ij}} \prod_j D_j^{\delta_{ij}}
\end{equation}

当$\beta_i (\beta_i + 1) \geq 0$时,$\frac{\partial^2 D_i}{\partial m_i^2} \geq 0$,$D_i$为凸函数;当$\beta_i (\beta_i + 1) < 0$时,$\frac{\partial^2 D_i}{\partial m_i^2} < 0$,$D_i$为凹函数。

故$D_i$在$\beta_i \geq 0$或$\beta_i \leq -1$时为凸函数,在$-1 < \beta_i < 0$时为凹函数。

\textbf{定理得证}
优化问题为凸优化问题当且仅当所有$D_i$均为凸函数。具体而言,当$\beta_i (\beta_i + 1) \geq 0$(即$\beta_i \geq 0$或$\beta_i \leq -1$)时满足凸性条件;当$-1 < \beta_i < 0$时问题为非凸。

\section{附录:附加图表}
\label{sec:appendix_figures}

\begin{figure}[H]
    \centering
    \includegraphics[width=1\textwidth]{图片10.png}
    \caption{730天窗口销量预测结果。(A)-(F)依次为花菜类、叶菜类、椒类、茄果类、食用菌类、水生块根类}
    \label{fig:fig10}
\end{figure}

\begin{figure}[H]
    \centering
    \includegraphics[width=1\textwidth]{图片11.png}
    \caption{730天窗口进货价格预测结果。(A)-(F)品类顺序同图\ref{fig:fig10}}
    \label{fig:fig11}
\end{figure}

\begin{figure}[H]
    \centering
    \includegraphics[width=1\textwidth]{图片12.png}
    \caption{365天窗口销量预测结果。(A)-(F)品类顺序同图\ref{fig:fig10}}
    \label{fig:fig12}
\end{figure}

\begin{figure}[H]
    \centering
    \includegraphics[width=1\textwidth]{图片13.png}
    \caption{365天窗口进货价格预测结果。(A)-(F)品类顺序同图\ref{fig:fig10}}
    \label{fig:fig13}
\end{figure}
\end{document}
